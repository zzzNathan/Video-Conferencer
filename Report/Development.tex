\pagestyle{fancy}
\rhead{\bfseries OCR A-Level Computer Science}
\chead{\mdseries \thepage}
\lhead{\bfseries Jonathan Kasongo \mdseries — May 2025/26}
\lfoot{\sffamily Candidate number: N/A}
\rfoot{\sffamily Centre number: N/A}

\chapter{Development}
\label{chap:development}

% Provided evidence of each stage of the iterative
% development process for a coded solution
% relating this to the break down of the problem
% from the analysis stage and explaining what they
% did and justifying why.

% Well structured and modular, code annotated, DRY

% Validation evidence 

\section{Iteration 1}

During this iteration I wanted to finish up designs for each 
of the pages that will appear on our website.

\subsubsection{Home page}

\begin{figure}[H]
\centering

\includegraphics[scale=0.2]{Images/HomeUI_2.png}

\caption{2nd Mock up of the home page.}
\label{fig:ui2}
\end{figure}

\begin{figure}[H]
\centering

\includegraphics[scale=0.2]{Images/HomeUI_3.png}

\caption{3rd Mock up of the home page.}
\label{fig:ui3}
\end{figure}

I started by cleaning up the mock design for our home page.
The first mock up was a very basic outline for how I wanted to 
style the page, so in the next couple iterations I polished 
the design and added more designs to fill the web page. These
additions made the home page feel more complete and less 
bare-bones, consequently making the webpage feel more 
professional for the user. \\ \vspace{0.2cm}

The next step was to transfer this design into html and css. 
Since my designs made use of a collection of frequently used 
colours I concluded that it would be useful to be able to use
these colours as defined variables instead of having to write 
out their hex colour codes each time 
(see section \ref{sec:ui}). However the native CSS syntax 
looks something like this. (Taken from \url{https://www.w3schools.com/css/css3_variables.asp})

\begin{minted}[linenos, bgcolor=lightestgray]{css}
:root {
  --blue: #1e90ff;
  --white: #ffffff;
}
\end{minted}

I personally found this syntax particularly disgusting, 
after doing some research I came across this stack overflow 
answer \url{https://stackoverflow.com/a/1877358}. After digging
into exactly what "sass" and "less" were I decided to use sass
as an extension of my CSS because I liked the look of it's 
syntax better. Sass (\textbf{S}yntactically \textbf{A}wesome 
\textbf{S}tylesheet) is a CSS extension language that 
provides a new syntax for writing CSS as well as a number of 
features that prevent repetition in the code like variables, 
functions and more. The sass code is saved in a \texttt{.sass}
file then one can compile the sass code into CSS by using the
command \texttt{sass <sass file name> <output css file name>}.
\\ \vspace{0.2cm}

After blocking in the basic elements of our webpage the HTML 
and sass code looked like this.

\textit{main.html}

\begin{minted}[linenos, bgcolor=lightestgray]{html}
<!DOCTYPE html>
<html>

<head>
  <meta charset="UTF-8">
  <meta name="viewport" content="width=device-width, initial-scale=1.0">
  <link href='https://fonts.googleapis.com/css?family=Inter' rel='stylesheet'>
  <link rel="stylesheet" href="styles.css">
  <title>Video-Conferencer</title>
</head>

<body>
  <div class="Headlines">

    <div class="Main_Headline">
      Video conferencing...
    </div>

    <div class="Sub_Headline">
      like you've never seen before.
    </div>

  </div>
</body>

</html>
\end{minted}

\textit{styles.sass}

\begin{minted}[linenos, bgcolor=lightestgray]{sass}
// Colour definitions
$Col_Main:       #69ca95
$Col_Secondary:  #284333
$Col_Tertiary:   #9a6442
$Col_Accent:     #ffc6e5
$Col_AccentDark: #c49df2

// Setting font and background colour
body
  background-color: $Col_Main
  font-family:      'Inter', sans-serif

// Styling for the headlines
.Headlines
  display:         flex
  flex-direction:  column
  align-items:     center
  justify-content: center

// Styling for the main headline
.Main_Headline
  width:       701px
  height:      238px
  font-size:   96px
  font-weight: 600 
  position:    absolute
  top:         20%
  left:        45%

// Styling for the sub headline
.Sub_Headline
  width:       603px
  height:      101px
  font-weight: 500
  text-align:  center
  font-size:   40px
  position:    absolute
  top:         60%
  left:        48%
\end{minted}

\subsubsection{Login page}
