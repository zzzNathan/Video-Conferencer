\pagestyle{fancy} \rhead{\bfseries OCR A-Level Computer Science}
\chead{\mdseries \thepage}
\lhead{\bfseries Jonathan Kasongo \mdseries — May 2025/26}
\lfoot{\sffamily Candidate number: N/A}
\rfoot{\sffamily Centre number: N/A}

\chapter{Evaluation}
\label{chap:evaluation}

\section{Post-development testing}

Since we have a few features that I didn't consider when I created the original post-development test plan we begin
by creating a revised post-development test plan.

\subsubsection{Test 0}

{\sffamily Task:} Access documentation site. \\

{\color{gray} \hrulefill}

{\sffamily Test type: Normal.}

\begin{enumerate}
  \item Navigate to the site.
  \item Click on Docs.
\end{enumerate}

{\sffamily Expected outcome:} Documentation should load successfully. \\

{\color{gray} \hrulefill}

\subsubsection{Test 1}

{\sffamily Task:} Create an account.\\

{\color{gray} \hrulefill} \\

{\textit{Note that this test allows our user to see our login pages. This links to the usability feature
"Log in/Sign up page}}
\\ \vspace{0.2cm}

{\sffamily Test type: Normal.}

\begin{enumerate}
  \item Navigate to the sign up page.
  \item Enter an adequate username into the username field.
  \item Enter the password \texttt{if5\#@Jal} into the password field.
  \item Click on the sign up button.
\end{enumerate}

{\sffamily Expected outcome:} User account created successfully. \\

{\color{gray} \hrulefill}

{\sffamily Test type: Erroneous.}\\

\begin{enumerate}
\item Repeat the steps except use the password \texttt{AopS12v}.\\
\end{enumerate}

{\sffamily Expected outcome:} User account not created, display message saying password is too weak.\\

{\color{gray} \hrulefill}

{\sffamily Test type: Erroneous.} \\

\begin{enumerate}
\item Repeat the steps except use the password \texttt{123}.
\end{enumerate}

{\sffamily Expected outcome:} User account not created, display
message saying password is too weak.\\

{\color{gray} \hrulefill}

\vspace{0.2cm}

\subsubsection{Test 2}

{\sffamily Task:} Log in to your account.\\

{\color{gray} \hrulefill}

{\textit{Note that this test allows our user to see our login pages. This links to the usability feature
"Log in/Sign up page}} \\ \vspace{0.2cm}

{\sffamily Test type: Normal.}\\

\begin{enumerate}
  \item Navigate to the log in page.
  \item Enter your actual username into the username field.
  \item Enter your actual password into the password field.
  \item Click on the log in button.
\end{enumerate}

{\sffamily Expected outcome:} User logged in successfully. \\

{\color{gray} \hrulefill}

{\sffamily Test type: Erroneous.}\\

\begin{enumerate}
  \item Repeat the steps except using the wrong password.
\end{enumerate}

{\sffamily Expected outcome:} User not logged in,
message displaying "Wrong username or password!". \\

{\color{gray} \hrulefill}

{\sffamily Test type: Erroneous.}\\

\begin{enumerate}
  \item Repeat the steps except using the wrong username.
\end{enumerate}

{\sffamily Expected outcome:} User not logged in,
message displaying "Wrong username or password!". \\

{\color{gray} \hrulefill}

\vspace{0.2cm}

\subsubsection{Test 3}

{\sffamily Task:} Create an event.\\

{\color{gray} \hrulefill}

{\sffamily Test type: Normal.} \\

\begin{enumerate}
  \item Navigate to the events page \\
  \item Click on the create event button \\
  \item Create an event, then refresh the page to check whether the event saved. \\
\end{enumerate}

{\sffamily Expected outcome: } Event that was created should still be present after refreshing page. \\

{\color{gray} \hrulefill}

\subsubsection{Test 4}

{\sffamily Task:} Delete an event.\\

{\color{gray} \hrulefill}

{\sffamily Test type: Normal.} \\

\begin{enumerate}
  \item Navigate to the events page. \\
  \item Delete the event you just created. \\
  \item Refresh the page to verify the event was indeed deleted. \\
\end{enumerate}

{\sffamily Expected outcome: } Event was properly deleted, event doesn't show after refresh. \\

{\color{gray} \hrulefill}

\subsubsection{Test 5}

{\sffamily Task:} Start a video call.\\

{\color{gray} \hrulefill}

{\sffamily Test type: Normal.}\\

\begin{enumerate}
  \item Connect a webcamera and microphone to your device (if necessary).
  \item Navigate to the create call page.
  \item Click on the create call button.
\end{enumerate}

{\sffamily Expected outcome:} Unique call code generated successfully
and displayed to the user. User is able to view themselves in
the video call.\\

{\color{gray} \hrulefill}

\vspace{0.2cm}

\subsubsection{Test 6}

{\sffamily Task:} Join a video call.\\ \vspace{0.2cm}

\textit{This test requires 2 people. Both users should have a camera
and microphone connected to their device. Both users should have a
stable internet connection satisfying the requirements listed in section \ref{sec:hardware}}\\

{\color{gray} \hrulefill}

{\sffamily Test type: Normal.}\\

\begin{enumerate}
  \item Have person 1 complete the steps outlined in Test 5.
  \item Person 1 should share the unique call code with person 2.
  \item Person 2 should then navigate to the join call page and enter the code that they were given.
  \item Person 2 should click on the join call button.
\end{enumerate}

{\sffamily Expected outcome:} Both users successfully joined the video call
and can clearly see and hear each other.\\

{\color{gray} \hrulefill} \\

\subsubsection{Test 7}

{\sffamily Task:} Log out.\\

{\color{gray} \hrulefill}

{\sffamily Test type: Normal.}\\

\begin{enumerate}
  \item Navigate to your account profile.
  \item Click the log out button.
\end{enumerate}

{\sffamily Expected outcome:} User logged out. \\

{\color{gray} \hrulefill}

\vspace{0.2cm}

Once again once we complete the post-development test plan we will fill in the following feedback form

\begin{longtblr}[
  caption={Post-development feedback form.}
]{
  colspec={lXXX},
  row{1}={lightestgray}
}
  No & Question & Rating (1-10) & Additional comments \\
  1 & How easy was the task to complete? & ~ & ~ \\
  2 & How easy was it to navigate the site? & ~ & ~ \\
  3 & How would you rate the design of the site? & ~ & ~ \\
  4 & How would you rate the user experience of the site? & ~ & ~ \\
  5 & Any additional comments or suggestions & N/A & ~ \\
\end{longtblr}

{\sffamily Daniel:}\\ \vspace{0.2cm}

{\sffamily Evidence: \url{https://youtu.be/EZyeDKM6ZR8}} \\ \vspace{0.2cm}
{\sffamily Evidence: \url{https://youtu.be/QwnYLBzXy2Y}} \\ \vspace{0.2cm}
{\sffamily Evidence: \url{https://youtu.be/MS5u3zo47N8}} \\ \vspace{0.2cm}

Here is Daniel's feedback form: \\ \vspace{0.2cm}

\begin{longtblr}[
  caption={Post-development feedback form.}
]{
  colspec={lXlX},
  row{1}={lightestgray}
}
  No & Question & Rating (1-10) & Additional comments \\
  1 & How easy was the task to complete? & 8 & Very easy as the layout was very clear. However I would suggest tutorial pop ups under under some icons on the first time through when someone makes an account so they can see where everything is. \\
  2 & How easy was it to navigate the site? & 9 & Easy and friendly to the user \\
  3 & How would you rate the design of the site? & 9 & I personally really like the design, it's very professional and unique\\
  4 & How would you rate the user experience of the site? & 10 & The site does exactly what I expected and was quite seamless, I wouldn't expect any other user to face issues with it either \\
  5 & Any additional comments or suggestions & N/A & I wouldn't comment on anything else other than it was very user friendly \\
\end{longtblr}

From our post-developement testing we have \textbf{clearly} provided annotated evidence of post development testing
for function and robustness. Note that by getting Daniel's feedback form and by analysing his comments we gain
invaluable feedback for our usability features. \\

\section{Evaluating success criteria}

\newcommand{\evalone}{%
This web page is indeed designed in an intuitive and user
friendly manner, this is demonstrated through
it's clear structure and visual hierarchy.
Users are immediately drawn to the prominent
gradient-styled heading "Video conferencing" which effectively
communicates the platform's core purpose. They then have option
to click one of the call to action buttons to either sign up or
login immediately after viewing the heading, if they so choose.
Finally the user is provided with a variety of common actions
in the top navigation bar, so they can quickly and efficiently
access the information they require.
\textit{Reference:} {\sffamily Test (16)} \url{https://youtu.be/Ii8QkbziWo4}
}

\newcommand{\evaltwo}{%
For our log in
system we make use of Clerk, a thoroughly tested and widely used
API for handling the user log in feature. Since Clerk is an
industry standard API we assume that their infastructure is indeed
secure and completely compliant with GDPR regulations.
\textit{Reference:} Refer to
\url{https://clerk.com/docs/security/overview} for details. \\ \vspace{0.1cm}

On the other hand for our events system though no errors arised during
iterative testing, we have not created a rigorous security test for
this part of the back end code, hence why I have rated this criterion
at 50\% completion. \\ \vspace{0.1cm}

This issue could be addressed in the future by using security
tools like ZAP which scan your code and look for potential
security vulnerabilities. To ensure safety an external cyber-security
company could be hired to perform a detailed analysis on the system,
from which all of the security vulnerabilities could be dealt with.
}

\newcommand{\evalthree}{%
Though this success criteria point is completely subjective, I do believe that the designs ended up being professional
looking and aesthetically appealing. This claim is reinforced by the results of the survey we conducted. Indeed >89\%
of our surveyees said they like the design fullfilling the criteria stated in success criteria 3. This shows that my
opinion is the majority and many others also find the UI to be professional looking and aesthetically appealing.
\textit{Reference:} {\sffamily Test (16)} \url{https://youtu.be/Ii8QkbziWo4}
}

\newcommand{\evalfour}{%
I have met this success criteria completely by using the 100ms library to allow users to video conference with each
other. As was discussed in the development section I tried a few different methods before landing on 100ms, however
I believe that the solution I have found right now provides the simplest way of implementing video conferencing. The
backend simply makes an HTTP request to the 100ms servers to ask for host and guest codes to a room, once we have
recieved these codes we display a dialog box to the user informing them of the code that they can then share to others.
Other users then simply navigate to the join call page, enter the code and they will have joined the video conference.
This feature is indeed evidenced in iterative test 15 as seen below.
\textit{Reference:} {\sffamily Test (15)} \url{https://youtu.be/mX33jG6B1Ik} \url{https://youtu.be/D2J8Y2TnWjo}
}

\newcommand{\evalfive}{%
I have met this success criteria completely by using the Clerk API to handle our login system. By importing the
Clerk library and retrieving the API key using an environment variable we use Clerk's built-in login and registration
components to provide a modern and technically robust username and password system.
\textit{Reference:} {\sffamily Test (2)} \url{https://youtu.be/IjwQH1Z8aPw}
}

\newcommand{\evalsix}{%
I believe that we have partially met this success criterion. We have used Clerk to manage usernames and passwords
in our application. From our tests in the development section, Clerk seems to only verify whether or not the given
password is 8 or more characters. For some very common passwords Clerk prohibits the user from trying to make an
account and gives the output \texttt{This password has been found as part of a breach and can not be used, please
try another password instead.} However Clerk doesn't give any specific password requirements other than
\texttt{Your password must contain 8 or more characters.} \textit{Reference:} {\sffamily Test (2)}
\url{https://youtu.be/p-rWgn692KU} \\ \vspace{0.1cm}

To solve this issue we have a number of options. On the buisness plan Clerk actually allows us to enforce specific
password requirements (\url{https://clerk.com/blog/a-new-password-experience}), however this will cost us money and
we currently have no way to make any money back. We could also look into and reseach other username and password
management libraries in order to find alternatives, this will ensure that our username and password will still be
robust and secure, through usage of a thoroughly tested and reputable Javascript library. Finally if we have no other
solutions one could resort to implementing the username and password system themselves from scratch, however the
downside with this approach is that it may not be \textit{completely} secure, without comprehensive security testing,
this fact coupled with the truth that implementing such a system will take a good amount of time and resources
justifies it being the last resort option.
}

\newcommand{\evalseven}{%
I believe that we have not met this success criteria. For our video conferencing system we utilise the 100ms video
conferencing library, which means that any latency issues will not be resolvable on our end. In this manner the
latency of our video conferencing applciation is totally dependent on the perfomance of the 100ms servers at that
moment in time. Since we have no control over the performance of the 100ms servers we cannot ever be certain that
our video conferencing latency won't exceed 150ms, because the performance of the 100ms servers will be constantly
changing and varying due to a number of factors outside of our control. The fact that we cannot ever affirmatively
make the claim that our latency is below 150ms coupled with the fact that no real attempt was made to test the
latency of our application, constructs a detailed justification as to why I believe that success criteria 7 was not
met. \\ \vspace{0.1cm}

We now propose an idea to potentially resolve this issue. Since the amount of latency will be constantly varying it
makes sense to conduct tests on the performance of our system at regular intervals (perhaps each hour). To achieve
this we could write a script that accesses the website and starts a video conference with 2 bots and computes a
value for the average latency over a span of some $n$ seconds. Occasionally the 100ms servers will be down for
maintenance or due to an outage. In such a case we could notify our user on the home and landing pages of the status
of the 100ms servers.
}

\newcommand{\evaleight}{%
I believe that we have partially achieved success criteria 8. The explanation of the success criteria explicitly
states that we will measure the success of this criterion by conducting a test with 6 different participants, however
during our iterative and post-developement testing we only conducted tests with at most 2 different participants. The
system should be able to theoretically handle 6 participants easily, however due to various time and resource
constraints I was unable to conduct the test with 6 participants. We know that the system should theoretically be able
to handle $\geq$ 6 participants by analysing the following page and looking at the free tier specifically
\url{https://www.100ms.live/pricing}, moreover we have evidence to show that video conferencing with 2 people was
smooth and worked without issues. \textit{Reference: See Daniel's post-development testing.}\\ \vspace{0.1cm}

We now propose a potential resolution for this issue. We could ask some friends or family to try and help conduct this
test but aligning the schedules of 6 people for this 1 test could be tough. If this approach doesn't work we could
potentially create an online posting looking for testers, and give them some monetary incentive if they accept to be a
part of the testing.
}

\newcommand{\evalnine}{%
I believe that we have partially completed success criterion 9. In a similar manner to the previous criterion, the
system should theoretically be able to handle 500 user accounts. We know this by analysing this page specifically at
the free tier \textit{Reference:} \url{https://clerk.com/pricing}. Indeed according to this page we should be able to
handle 10000 monthly user accounts. However we have failed to conduct proper rigorous tests in order to acertain
whether or not system performance degrades after a certain number of users. \\ \vspace{0.1cm}

We now propose a potential solution to this issue. Getting 500 actual users to register an account could be
logistically difficult, so alternatively we could write a script that makes bots to register accounts on our website
programatically.
}

\begin{longtblr}[
  caption={Evaluation of the success criteria.}
]{
  colspec={l X[2] X[6] r},
  row{1}={lightestgray},
  row{2}={lightestgray},
  row{8}={lightestgray},
  rowhead=1
}
No & Criterion & Evaluation & \% \\
{\sffamily Qualitative} & & & \\
1 & System should be intuitive and easy to grasp & \evalone & 100\% \\
2 & The user's data should be properly secured & \evaltwo & 50\% \\
3 & The webpage design should be aesthetically pleasing & \evalthree & 100\% \\
4 & Users should be able to create and join video conferences & \evalfour & 100\% \\
5 & Users should be able to create, login and logout of their own accounts & \evalfive & 100\% \\
{\sffamily Quantitative} & & & \\
6 & User passwords should be of an adequate strength & \evalsix & 50\% \\
7 & The video call latency should not exceed 150ms & \evalseven & 0\% \\
8 & The system should be able to handle video conferences with more than 2 participants & \evaleight & 50\% \\
9 & The system should be able to store at least 500 user accounts & \evalnine & 50\% \\
\end{longtblr}

\section{Evaluation of the usability features}

\newcommand{\usaone}{%
I believe that we have completed this feature completely, as we recall designing and implementing
the home page in the development section. The final design uses React and JSX to render the front end.
\textit{Reference:} \sffamily{Test (2)}
}

\newcommand{\usatwo}{%
I believe that we have completed this feature completely, as we have designed and implemented
a log in page as well as a sign up page in our development section. The final design makes use of the Clerk SDK to
hanle the registration and login logic, and the front end is implemented with React. \textit{Reference:}
\sffamily{Test (3)}
}

\newcommand{\usathree}{%
I believe that we have not completed this feature. Due to time constraints we unfortunately
could not get to creating a helper document. \\ \vspace{0.1cm} We note that this issue is rather easy to fix, since
we could simply ask the IT staff at my client's workplace to create the document. This will fully resolve the issue.
}

\newcommand{\usafour}{%
I believe that we have not completed this feature. Due to time constraints we were not able to implement
this feature. \\ \vspace{0.1cm} Implementing macros inside the video conferencing system is not a trivial task at all,
since we are making use of the 100ms library. From my understanding the 100ms library
}

\newcommand{\usafive}{%
I believe that we have not completed this feature. Due to time constraints we were not able to
implement this feature.
}

% Give evidence via video, commentate them with usability features
\begin{longtblr}[
  caption={Evaluation of the usability features.}
]{
  colspec={l X[2] X[6] r},
  row{1}={lightestgray},
  rowhead=1
}
No & Feature & Evaluation & \% \\
1 & Home page & \usaone & 100\% \\

2 & Log in/Sign up page & \usatwo & 100\% \\

3 & Help PDF document & \usathree & 0\% \\

4 & Macros & \usafour & 0\% \\

5 & Video tutorial & \usafive & 0\% \\
\end{longtblr}

\section{Limitations}

We begin by discussing the limitations that we considered in our analysis section \ref{sec:limitations}.

\subsection{Initial limitations}

\subsubsection{Internet}

Indeed as we explored in our analysis an internet connection is necessary for the user to be able to access our system.
As previously discussed we notice that the requirement of having a stable internet connection could potentially
restrict some users with poor economic situations from using our system. Indeed this opposes the initial motivation
for this project: to create an \textit{accessible} video conferencing application. \\ \vspace{0.2cm}

Though the requirement to have a stable internet connection has it's negatives, we argue that it is necessary since
there are no realistic solutions to this limitation. Indeed in order to conduct video conferencing users will have to
send video and/or audio data to the other users in the conference one way or another, there is no arguing this fact.
In this time period \footnote{At the time of writing the year is currently 2025} we have 2 main methods of data
communication. The first is to send signals over a WAN like the internet. The second is to physically connect devices
together in a LAN. If the first option is off the table we can only employ the second option. We note that the second
option requires a much more tedious setup as well as requires each user to be in relatively close proximity to all
the other users in the video conference. These inconveniences far outweigh the inconveniece of having to purchase and
install a stable internet connection.

\subsubsection{Passwords}

We previously discussed the issue of users forgetting their own passwords and getting logged out of their accounts.
Though we understand that we can never fix this issue totally, we argue that we have derived a partial solution.
\\ \vspace{0.2cm}

In using the Clerk library in order to handle our user management systems, we note that they allow our users to be
able to login to our system using their google accounts. Using their google accounts the user doesn't need to remember
their password as they may simply click on the "Continue with Google" button after which they will be able to easily
sign in using their Gmail. \\ \vspace{0.2cm}

\subsubsection{Signalling servers}

When we initially wrote the limitations section, I assumed that I would be working with WebRTC library in order to
facilitate the implementation of video conferencing in our application. However in the final solution we note that we
instead make use of the 100ms video conferencing library. Not only does this library provide a much simpler method
to implement video conferencing it also renders the system much more maintainable as our maintainers will have a lot
less work to complete if the system isn't too complex. \\ \vspace{0.2cm}

With that it is easy to see that this limitation no longer applies to us.

\subsubsection{Social engineering}

We construct a philosophical arguement to justify the claim that we can never truly guarentee that this issue is
resolved. Due to human nature we understand that in the 21st century we can never completely eliminate the effect of
evil. Robert Greene argues in his book \textit{The 48 Laws of Power} that humans act only for their own benefit
\cite{power}, so if a situation arises in which someone obtains sufficient motivation to commit an act of evil,
in this case the use of social engineering to obtain somebody else's password, then in their eyes such an act doesn't
become morally injust anymore. Thomas Hobbes reinforces this school of thought in his book \textit{Leviathan} \cite{lev}.
He argues that man is naturally self-interested, implying that what we might think of as evil could be an inevitable
result of man's struggle for survival. Moreover we note that every human sees the world differently and consequently
has their own morals and views as a result of their experiences. As man wanders through life and encounters new
experiences his thoughts and morals will constantly morph and shift. In this manner our struggle against evil is not
a discrete one-time battle but is rather a continous everlasting commitment to fighting for one's own views of morally
just and injust.
\\ \vspace{0.2cm}

In order stop humans from commiting such acts we would have to monitor and restrict the actions of each and every one
of the $\approx$ 8 million people on earth. Such drastic measures would not only be \textbf{extremely} resource
intensive for my client and his small team but would also be morally dubious to the point where it could perhaps
illicit global-scale actions in response. This reasoning alone is enough to justify the fact that this limitation can
never \textbf{truly} be fixed. \\ \vspace{0.2cm}

We now discuss some new limitations that were not considered in the analysis section.

\subsection{New limitiations}

\subsubsection{Helper document}

Unfortunately we were unable to create the helper document that we spoke of in section \ref{sec:usability} due to
time constraints in our development sections. Since we lack a \textit{user guide} it may be difficult for new users
to understand where everything is and what features exist in the system, this was highlighted in Daniel's critique:
\textit{"I would suggest tutorial pop ups under
under some icons on the first time through
when someone makes an account so they
can see where everything is."} However we hope that the simple design of our system compensates for the
lack of a helper document as most users should be able to understand how to use and navigate the software, this
conclusion is justified by the feedback we recieved from Daniel during our post-development testing, where he
explicity mentioned how easy the software was to use: \textit{"Very easy as the layout was very clear."}. \\
\vspace{0.2cm}

Thankfully this limitation can easily be remedied by asking one of the IT staff at our client's place of work to create
a helper document, and then implement a button on the navbar that allows the users to click and have the helper
document open up in another tab.

\subsubsection{Landing page}

The final landing page design ended up being simple and professional looking, however there are still improvements
to be made. The landing page should demonstrate some of the core features of our application whilst remaining
professional looking, modern and eye-catching.

\section{Maintenance}
