\pagestyle{fancy} \rhead{\bfseries OCR A-Level Computer Science}
\chead{\mdseries \thepage}
\lhead{\bfseries Jonathan Kasongo \mdseries — May 2025/26}
\lfoot{\sffamily Candidate number: N/A}
\rfoot{\sffamily Centre number: N/A}

\chapter{Evaluation}
\label{chap:evaluation}

\section{Post-development testing}

Since we have a few features that I didn't consider when I created the original post-development test plan we begin
by creating a revised post-development test plan.

\subsubsection{Test 0}

{\sffamily Task:} Access documentation site. \\

{\color{gray} \hrulefill}

{\sffamily Test type: Normal.}

\begin{enumerate}
  \item Navigate to the site.
  \item Click on Docs.
\end{enumerate}

{\sffamily Expected outcome:} Documentation should load successfully. \\

{\color{gray} \hrulefill}

\subsubsection{Test 1}

{\sffamily Task:} Create an account.\\

{\color{gray} \hrulefill}

{\sffamily Test type: Normal.}

\begin{enumerate}
  \item Navigate to the sign up page.
  \item Enter an adequate username into the username field.
  \item Enter the password \texttt{if5\#@Jal} into the password field.
  \item Click on the sign up button.
\end{enumerate}

{\sffamily Expected outcome:} User account created successfully. \\

{\color{gray} \hrulefill}

{\sffamily Test type: Erroneous.}\\

\begin{enumerate}
\item Repeat the steps except use the password \texttt{AopS12v}.\\
\end{enumerate}

{\sffamily Expected outcome:} User account not created, display message saying password is too weak.\\

{\color{gray} \hrulefill}

{\sffamily Test type: Erroneous.} \\

\begin{enumerate}
\item Repeat the steps except use the password \texttt{123}.
\end{enumerate}

{\sffamily Expected outcome:} User account not created, display
message saying password is too weak.\\

{\color{gray} \hrulefill}

\vspace{0.2cm}

\subsubsection{Test 2}

{\sffamily Task:} Log in to your account.\\

{\color{gray} \hrulefill}

{\sffamily Test type: Normal.}\\

\begin{enumerate}
  \item Navigate to the log in page.
  \item Enter your actual username into the username field.
  \item Enter your actual password into the password field.
  \item Click on the log in button.
\end{enumerate}

{\sffamily Expected outcome:} User logged in successfully. \\

{\color{gray} \hrulefill}

{\sffamily Test type: Erroneous.}\\

\begin{enumerate}
  \item Repeat the steps except using the wrong password.
\end{enumerate}

{\sffamily Expected outcome:} User not logged in,
message displaying "Wrong username or password!". \\

{\color{gray} \hrulefill}

{\sffamily Test type: Erroneous.}\\

\begin{enumerate}
  \item Repeat the steps except using the wrong username.
\end{enumerate}

{\sffamily Expected outcome:} User not logged in,
message displaying "Wrong username or password!". \\

{\color{gray} \hrulefill}

\vspace{0.2cm}

\subsubsection{Test 3}

{\sffamily Task:} Create an event.\\

{\color{gray} \hrulefill}

{\sffamily Test type: Normal.} \\

\begin{enumerate}
  \item Navigate to the events page \\
  \item Click on the create event button \\
  \item Create an event, then refresh the page to check whether the event saved. \\
\end{enumerate}

{\sffamily Expected outcome: } Event that was created should still be present after refreshing page. \\

{\color{gray} \hrulefill}

\subsubsection{Test 4}

{\sffamily Task:} Delete an event.\\

{\color{gray} \hrulefill}

{\sffamily Test type: Normal.} \\

\begin{enumerate}
  \item Navigate to the events page. \\
  \item Delete the event you just created. \\
  \item Refresh the page to verify the event was indeed deleted. \\
\end{enumerate}

{\sffamily Expected outcome: } Event was properly deleted, event doesn't show after refresh. \\

{\color{gray} \hrulefill}

\subsubsection{Test 5}

{\sffamily Task:} Start a video call.\\

{\color{gray} \hrulefill}

{\sffamily Test type: Normal.}\\

\begin{enumerate}
  \item Connect a webcamera and microphone to your device (if necessary).
  \item Navigate to the create call page.
  \item Click on the create call button.
\end{enumerate}

{\sffamily Expected outcome:} Unique call code generated successfully
and displayed to the user. User is able to view themselves in
the video call.\\

{\color{gray} \hrulefill}

\vspace{0.2cm}

\subsubsection{Test 6}

{\sffamily Task:} Join a video call.\\ \vspace{0.2cm}

\textit{This test requires 2 people. Both users should have a camera
and microphone connected to their device. Both users should have a
stable internet connection satisfying the requirements listed in section \ref{sec:hardware}}\\

{\color{gray} \hrulefill}

{\sffamily Test type: Normal.}\\

\begin{enumerate}
  \item Have person 1 complete the steps outlined in Test 5.
  \item Person 1 should share the unique call code with person 2.
  \item Person 2 should then navigate to the join call page and enter the code that they were given.
  \item Person 2 should click on the join call button.
\end{enumerate}

{\sffamily Expected outcome:} Both users successfully joined the video call
and can clearly see and hear each other.\\

{\color{gray} \hrulefill} \\

\subsubsection{Test 7}

{\sffamily Task:} Log out.\\

{\color{gray} \hrulefill}

{\sffamily Test type: Normal.}\\

\begin{enumerate}
  \item Navigate to your account profile.
  \item Click the log out button.
\end{enumerate}

{\sffamily Expected outcome:} User logged out. \\

{\color{gray} \hrulefill}

\vspace{0.2cm}

Once again once we complete the post-development test plan we will fill in the following feedback form

\begin{longtblr}[
  caption={Post-development feedback form.}
]{
  colspec={lXXX},
  row{1}={lightestgray}
}
  No & Question & Rating (1-10) & Additional comments \\
  1 & How easy was the task to complete? & ~ & ~ \\
  2 & How easy was it to navigate the site? & ~ & ~ \\
  3 & How would you rate the design of the site? & ~ & ~ \\
  4 & How would you rate the user experience of the site? & ~ & ~ \\
  5 & Any additional comments or suggestions & N/A & ~ \\
\end{longtblr}

{\sffamily Daniel:}\\ \vspace{0.2cm}

{\sffamily Evidence: \url{https://youtu.be/EZyeDKM6ZR8}} \\ \vspace{0.2cm}
{\sffamily Evidence: \url{https://youtu.be/QwnYLBzXy2Y}} \\ \vspace{0.2cm}
{\sffamily Evidence: \url{https://youtu.be/MS5u3zo47N8}} \\ \vspace{0.2cm}

Here is Daniel's feedback form: \\ \vspace{0.2cm}

\begin{longtblr}[
  caption={Post-development feedback form.}
]{
  colspec={lXlX},
  row{1}={lightestgray}
}
  No & Question & Rating (1-10) & Additional comments \\
  1 & How easy was the task to complete? & 8 & Very easy as the layout was very clear. However I would suggest tutorial pop ups under under some icons on the first time through when someone makes an account so they can see where everything is. \\
  2 & How easy was it to navigate the site? & 9 & Easy and friendly to the user \\
  3 & How would you rate the design of the site? & 9 & I personally really like the design, it's very professional and unique\\
  4 & How would you rate the user experience of the site? & 10 & The site does exactly what I expected and was quite seamless, I wouldn't expect any other user to face issues with it either \\
  5 & Any additional comments or suggestions & N/A & I wouldn't comment on anything else other than it was very user friendly \\
\end{longtblr}

From our post-developement testing we have \textbf{clearly} provided annotated evidence of post development testing
for function and robustness. \\

\section{Evaluating success criteria}

\newcommand{\evalone}{%
This web page is indeed designed in an intuitive and user
friendly manner, this is demonstrated through
it's clear structure and visual hierarchy.
Users are immediately drawn to the prominent
gradient-styled heading "Video conferencing" which effectively
communicates the platform's core purpose. They then have option
to click one of the call to action buttons to either sign up or
login immediately after viewing the heading, if they so choose.
Finally the user is provided with a variety of common actions
in the top navigation bar, so they can quickly and efficiently
access the information they require.
\textit{Reference:} {\sffamily Test (16)} \url{https://youtu.be/Ii8QkbziWo4}
}

\newcommand{\evaltwo}{%
For our log in
system we make use of Clerk, a thoroughly tested and widely used
API for handling the user log in feature. Since Clerk is an
industry standard API we assume that their infastructure is indeed
secure and completely compliant with GDPR regulations.
\textit{Reference:} Refer to
\url{https://clerk.com/docs/security/overview} for details. \\ \vspace{0.1cm}

On the other hand for our events system though no errors arised during
iterative testing, we have not created a rigorous security test for
this part of the back end code, hence why I have rated this criterion
at 50\% completion. \\ \vspace{0.1cm}

This issue could be addressed in the future by using security
tools like ZAP which scan your code and look for potential
security vulnerabilities. To ensure safety an external cyber-security
company could be hired to perform a detailed analysis on the system,
from which all of the security vulnerabilities could be dealt with.
}

\newcommand{\evalthree}{%
Though this success criteria point is completely subjective, I do believe that the designs ended up being professional
looking and aesthetically appealing. This claim is reinforced by the results of the survey we conducted. Indeed >89\%
of our surveyees said they like the design fullfilling the criteria stated in success criteria 3. This shows that my
opinion is the majority and many others also find the UI to be professional looking and aesthetically appealing.
\textit{Reference:} {\sffamily Test (16)} \url{https://youtu.be/Ii8QkbziWo4}
}

\newcommand{\evalfour}{%
I have met this success criteria completely by using the 100ms library to allow users to video conference with each
other. As was discussed in the development section I tried a few different methods before landing on 100ms, however
I believe that the solution I have found right now provides the simplest way of implementing video conferencing. The
backend simply makes an HTTP request to the 100ms servers to ask for host and guest codes to a room, once we have
recieved these codes we display a dialog box to the user informing them of the code that they can then share to others.
Other users then simply navigate to the join call page, enter the code and they will have joined the video conference.
This feature is indeed evidenced in iterative test 15 as seen below.
\textit{Reference:} {\sffamily Test (15)} \url{https://youtu.be/mX33jG6B1Ik} \url{https://youtu.be/D2J8Y2TnWjo}
}

\newcommand{\evalfive}{%
I have met this success criteria completely by using the Clerk API to handle our login system. By importing the
Clerk library and retrieving the API key using an environment variable we use Clerk's built-in login and registration
components to provide a modern and technically robust username and password system.
\textit{Reference:} {\sffamily Test (2)} \url{https://youtu.be/IjwQH1Z8aPw}
}

\newcommand{\evalsix}{%
I believe that we have partially met this success criterion. We have used Clerk to manage usernames and passwords
in our application. From our tests in the development section, Clerk seems to only verify whether or not the given
password is 8 or more characters. For some very common passwords Clerk prohibits the user from trying to make an
account and gives the output \texttt{This password has been found as part of a breach and can not be used, please
try another password instead.} However Clerk doesn't give any specific password requirements other than
\texttt{Your password must contain 8 or more characters.} \textit{Reference:} {\sffamily Test (2)}
\url{https://youtu.be/p-rWgn692KU} \\ \vspace{0.1cm}

To solve this issue we have a number of options. On the buisness plan Clerk actually allows us to enforce specific
password requirements (\url{https://clerk.com/blog/a-new-password-experience}), however this will cost us money and
we currently have no way to make any money back. We could also look into and reseach other username and password
management libraries in order to find alternatives, this will ensure that our username and password will still be
robust and secure, through usage of a thoroughly tested and reputable Javascript library. Finally if we have no other
solutions one could resort to implementing the username and password system themselves from scratch, however the
downside with this approach is that it may not be \textit{completely} secure, without comprehensive security testing,
this fact coupled with the truth that implementing such a system will take a good amount of time and resources
justifies it being the last resort option.
}

\newcommand{\evalseven}{%
I believe that we have not met this success criteria. For our video conferencing system we utilise the 100ms video
conferencing library, which means that any latency issues will not be resolvable on our end. In this manner the
latency of our video conferencing applciation is totally dependent on the perfomance of the 100ms servers at that
moment in time. Since we have no control over the performance of the 100ms servers we cannot ever be certain that
our video conferencing latency won't exceed 150ms, because the performance of the 100ms servers will be constantly
changing and varying due to a number of factors outside of our control. The fact that we cannot ever affirmatively
make the claim that our latency is below 150ms coupled with the fact that no real attempt was made to test the
latency of our application, constructs a detailed justification as to why I believe that success criteria 7 was not
met. \\ \vspace{0.1cm}

We now propose an idea to potentially resolve this issue. Since the amount of latency will be constantly varying it
makes sense to conduct tests on the performance of our system at regular intervals (perhaps each hour). To achieve
this we could write a script that accesses the website and starts a video conference with 2 bots and computes a
value for the average latency over a span of some $n$ seconds. Occasionally the 100ms servers will be down for
maintenance or due to an outage. In such a case we could notify our user on the home and landing pages of the status
of the 100ms servers.
}

\begin{longtblr}{
  colspec={lXXr},
  row{1}={lightestgray},
  row{2}={lightestgray}
}
No & Criterion & Evaluation & \% \\
{\sffamily Qualitative} & & & \\
1 & System should be intuitive and easy to grasp & \evalone & 100\% \\
2 & The user's data should be properly secured & \evaltwo & 50\% \\
3 & The webpage design should be aesthetically pleasing & \evalthree & 100\% \\
4 & Users should be able to create and join video conferences & \evalfour & 100\% \\
5 & Users should be able to create, login and logout of their own accounts & \evalfive & 100\% \\
6 & User passwords should be of an adequate strength & \evalsix & 50\% \\
7 & The video call latency should not exceed 150ms & \evalseven & 0\% \\
\end{longtblr}
