\pagestyle{fancy} \rhead{\bfseries OCR A-Level Computer Science}
\chead{\mdseries \thepage}
\lhead{\bfseries Jonathan Kasongo \mdseries — May 2025/26}
\lfoot{\sffamily Candidate number: N/A}
\rfoot{\sffamily Centre number: N/A}

\chapter{Evaluation}
\label{chap:evaluation}

\section{Post-development testing}

Since we have a few features that I didn't consider when I created the original post-development test plan we begin
by creating a revised post-development test plan.

\subsubsection{Test 1}

{\sffamily Task:} Create an account.\\

{\color{gray} \hrulefill}

{\sffamily Test type: Normal.}

\begin{enumerate}
  \item Navigate to the sign up page.
  \item Enter an adequate username into the username field.
  \item Enter the password \texttt{if5\#@Jal} into the password field.
  \item Click on the sign up button.
\end{enumerate}

{\sffamily Expected outcome:} User account created successfully. \\

{\color{gray} \hrulefill}

{\sffamily Test type: Extreme.}\\

\begin{enumerate}
\item Repeat the steps except use the password \texttt{AopS12v}.\\
\end{enumerate}

{\sffamily Expected outcome:} User account created successfully.\\

{\color{gray} \hrulefill}

{\sffamily Test type: Erroneous.} \\

\begin{enumerate}
\item Repeat the steps except use the password \texttt{123}.
\end{enumerate}

{\sffamily Expected outcome:} User account not created, display
message saying password is too weak.\\

{\color{gray} \hrulefill}

\vspace{0.2cm}

\subsubsection{Test 2}

{\sffamily Task:} Log in to your account.\\

{\color{gray} \hrulefill}

{\sffamily Test type: Normal.}\\

\begin{enumerate}
  \item Navigate to the log in page.
  \item Enter your actual username into the username field.
  \item Enter your actual password into the password field.
  \item Click on the log in button.
\end{enumerate}

{\sffamily Expected outcome:} User logged in successfully. \\

{\color{gray} \hrulefill}

{\sffamily Test type: Erroneous.}\\

\begin{enumerate}
  \item Repeat the steps except using the wrong password.
\end{enumerate}

{\sffamily Expected outcome:} User not logged in,
message displaying "Wrong username or password!". \\

{\color{gray} \hrulefill}

\vspace{0.2cm}

\subsubsection{Test 3}

{\sffamily Task:} Create an event.\\

{\color{gray} \hrulefill}

{\sffamily Test type: Normal.} \\

\begin{enumerate}
  \item Navigate to the events page \\
  \item Click on the create event button \\
  \item Create an event, then refresh the page to check whether the event saved. \\
\end{enumerate}

{\sffamily Expected outcome: } Event that was created should still be present after refreshing page. \\

{\color{gray} \hrulefill}

\subsubsection{Test 4}

{\sffamily Task:} Delete an event.\\

{\color{gray} \hrulefill}

{\sffamily Test type: Normal.} \\

\begin{enumerate}
  \item Navigate to the events page. \\
  \item Delete the event you just created. \\
  \item Refresh the page to verify the event was indeed deleted. \\
\end{enumerate}

{\sffamily Expected outcome: } Event was properly deleted, event doesn't show after refresh. \\

{\color{gray} \hrulefill}

\subsubsection{Test 5}

{\sffamily Task:} Start a video call.\\

{\color{gray} \hrulefill}

{\sffamily Test type: Normal.}\\

\begin{enumerate}
  \item Connect a webcamera and microphone to your device (if necessary).
  \item Navigate to the create call page.
  \item Click on the create call button.
\end{enumerate}

{\sffamily Expected outcome:} Unique call code generated successfully
and displayed to the user. User is able to view themselves in
the video call.\\

{\color{gray} \hrulefill}

\vspace{0.2cm}

\subsubsection{Test 6}

{\sffamily Task:} Join a video call.\\ \vspace{0.2cm}

\textit{This test requires 2 people. Both users should have a camera
and microphone connected to their device. Both users should have a
stable internet connection satisfying the requirements listed in section \ref{sec:hardware}}\\

{\color{gray} \hrulefill}

{\sffamily Test type: Normal.}\\

\begin{enumerate}
  \item Have person 1 complete the steps outlined in Test 5.
  \item Person 1 should share the unique call code with person 2.
  \item Person 2 should then navigate to the join call page and enter the code that they were given.
  \item Person 2 should click on the join call button.
\end{enumerate}

{\sffamily Expected outcome:} Both users successfully joined the video call
and can clearly see and hear each other.\\

{\color{gray} \hrulefill} \\

\subsubsection{Test 7}

{\sffamily Task:} Log out.\\

{\color{gray} \hrulefill}

{\sffamily Test type: Normal.}\\

\begin{enumerate}
  \item Navigate to your account profile.
  \item Click the log out button.
\end{enumerate}

{\sffamily Expected outcome:} User logged out. \\

{\color{gray} \hrulefill}

\vspace{0.2cm}

Once again once we complete the post-development test plan we will fill in the following feedback form

\begin{longtblr}[
  caption={Post-development feedback form.}
]{
  colspec={X[1]X[4]X[2]X[4]},
  row{1}={lightestgray}
}
  No & Question & Rating (1-10) & Additional comments \\
  1 & How easy was the task to complete? & & \\
  2 & How easy was it to navigate the site? & & \\
  3 & How would you rate the design of the site? & & \\
  4 & How would you rate the user experience of the site? & & \\
  5 & Any additional comments or suggestions & N/A & \\
\end{longtblr}

\section{Evaluating success criteria}

\newcommand{\evalone}{%
This web page is indeed designed in an intuitive and user
friendly manner, this is demonstrated through
it's clear structure and visual hierarchy.
Users are immediately drawn to the prominent
gradient-styled heading "Video conferencing" which effectively
communicates the platform's core purpose. They then have option
to click one of the call to action buttons to either sign up or
login immediately after viewing the heading, if they so choose.
Finally the user is provided with a variety of common actions
in the top navigation bar, so they can quickly and efficiently
access the information they require.
\textit{Reference:} {\sffamily Test (16)} \url{https://youtu.be/Ii8QkbziWo4}
}

\newcommand{\evaltwo}{%
For our log in
system we make use of Clerk, a thoroughly tested and widely used
API for handling the user log in feature. Since Clerk is an
industry standard API we assume that their infastructure is indeed
secure and compliant with GDPR. \textit{Reference:} Refer to
\url{https://clerk.com/docs/security/overview} for details. \\ \vspace{0.1cm}

On the other hand for our events system though no errors arised during
iterative testing, we have not created a rigorous security test for
this part of the back end code, hence why I have rated this criterion
at 50\% completion.
}

\newcommand{\evalthree}{%
Though this success criteria point is completely subjective, I do believe that the designs ended up being professional
looking and aesthetically appealing. This claim is reinforced by the results of the survey we conducted. Indeed >89\%
of our surveyees said they like the design fullfilling the criteria stated in success criteria 3.
\textit{Reference:} {\sffamily Test (16)} \url{https://youtu.be/Ii8QkbziWo4}
}

\newcommand{\evalfour}{%
I have met this success criteria completely by using the 100ms library to allow users to video conference with each
other. As was discussed in the development section I tried a few different methods before landing on 100ms, however
I believe that the solution I have found right now provides the simplest way of implementing video conferencing. The
backend simply makes an HTTP request to the 100ms servers to ask for host and guest codes to a room, once we have
recieved these codes we display a dialog box to the user informing them of the code that they can then share to others.
Other users then simply navigate to the join call page, enter the code and they will have joined the video conference.
\textit{Reference:} {\sffamily Test (15)}
}

\begin{longtblr}{
  colspec={lXXr},
  row{1}={lightestgray},
  row{2}={lightestgray}
}
No & Criterion & Evaluation & \% \\
{\sffamily Qualitative} & & & \\
1 & System should be intuitive and easy to grasp & \evalone & 100\% \\
2 & The user's data should be properly secured & \evaltwo & 50\% \\
3 & The webpage design should be aesthetically pleasing & \evalthree & 100\% \\
4 & Users should be able to create and join video conferences & \evalfour & 100\% \\
\end{longtblr}
