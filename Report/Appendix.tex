\appendix

\chapter{Code listings}

\section{Front end}

\underline{/src/main.jsx}

\begin{minted}[linenos, bgcolor=lightestgray]{jsx}
import { createRoot } from "react-dom/client"
import { createBrowserRouter, RouterProvider } from "react-router-dom"
import Landing from "./components/Landing.jsx"
import Registration from "./components/Registration.jsx"
import Login from "./components/Login.jsx"
import Home from "./components/Home.jsx"

// Path is an extension that goes after our URL,
// once this extension is written the corresponding
// React element will be loaded and rendered.
const router = createBrowserRouter([
  {
    path: "/",
    element: <Landing />
  },
  {
    path: "/registration",
    element: <Registration />
  },
  {
    path: "/login",
    element: <Login />
  },
  {
    path: "/home",
    element: <Home />
  }
])

createRoot(document.getElementById('root')).render(
  <RouterProvider router={router} />
)
\end{minted}

\underline{/src/Button.jsx}

\begin{minted}[linenos, bgcolor=lightestgray]{jsx}
import "../styles/Button.sass"

// Simple sign up button component
function Button()
{
  return (
    <button class="Button"> Sign Up </button>
  )
}

export default Button 
\end{minted}

\underline{/src/Clock.jsx}

\begin{minted}[linenos, bgcolor=lightestgray]{jsx}
import "../styles/Clock.sass"
import { useEffect, useState } from "react"

// Gets current date and time and continously updates it every second
function Clock ()
{
  const [currentDateTime, setCurrentDateTime] = useState("")

  const updateDateTime = () => {
    const now = new Date()

    // Get components of the date
    const day    = String(now.getDate()).padStart(2, "0")
    const month  = String(now.getMonth() + 1).padStart(2, "0")
    const year   = now.getFullYear()
    const hour   = String(now.getHours()).padStart(2, '0')
    const minute = String(now.getMinutes()).padStart(2, '0')
    const second = String(now.getSeconds()).padStart(2, '0')
    
    // Get the day name
    const options = { weekday: "long" }
    const dayName = now.toLocaleDateString("en-US", options).substring(0,3)

    // Format the date string
    setCurrentDateTime(`${day}-${month}-${year} ${dayName}\n
	                ${hour}:${minute}:${second}`)
  };

  useEffect(() => {
    updateDateTime()

    // Update every second
    const intervalId = setInterval(updateDateTime, 1000) 
    
    // Cleanup interval on unmount
    return () => clearInterval(intervalId)
  }, []);

  // Toggle the opacity of the clock when the user clicks on it
  const [opacity, setOpacity] = useState(1)
  const Toggle_Opacity = () => {
    setOpacity(prevOpacity => (prevOpacity === 0.1 ? 1 : 0.1))
  }
  
  // Render the clock
  return (
    <div
      className="Clock"
      onClick={Toggle_Opacity}
      style={{opacity: opacity}}>

      {currentDateTime}

    </div>
  )
}

export default Clock
\end{minted}

\underline{/src/Landing.jsx}

\begin{minted}[linenos, bgcolor=lightestgray]{jsx}
import { ClerkProvider } from "@clerk/clerk-react"
import Button from "./Button"
import Navbar from "./Navbar"
import Quotes from "../assets/Quote_Cards.svg"
import "../styles/Landing.sass"

const PUBLISHABLE_KEY = import.meta.env.VITE_CLERK_PUBLISHABLE_KEY

// Code to render the landing page of our site
function Landing () {
  return (
    <> 
    <ClerkProvider publishableKey={PUBLISHABLE_KEY}>
      <Navbar />
    
      <div className="Main_Graphic">
        <h1 className="Main_Headline"> Video conferencing... </h1>

        <h2 className="Sub_Headline"> like you've <i>
        <span className="Pink_Text"> never </span> </i> seen it before </h2>
      </div>

      <div className="Buttons">
        <a href="/login"><button className="Login_Button"> Login </button></a>
        <a href="/registration"><Button /></a>
      </div>

      <img className="Quote_Card" src={Quotes} /> 
    </ClerkProvider>
    </>
  )
}

export default Landing
\end{minted}

\underline{/src/Navbar.jsx}

\begin{minted}[linenos, bgcolor=lightestgray]{jsx}
import { useUser } from '@clerk/clerk-react'
import { FontAwesomeIcon } from "@fortawesome/react-fontawesome"
import { faCog } from "@fortawesome/free-solid-svg-icons"
import "../styles/Navbar.sass"

// Makes the navigation bar design a component must be inside
// a clerk provider
function Navbar()
{
  // If user isn't signed in then they can be redirected
  // to the registration page
  const { isSignedIn } = useUser()
  var link = "/registration"
  var text = "Get started"

  // If user is signed in then they can go straight
  // to the home page
  if (isSignedIn)
  {
    link = "/home"
    text = "Home"
  }
  
  return (
    <nav class="Navbar">
      <ul class="Navbar_List">
        <li class="Navbar_Item"> <b>  Video-Conferencer </b> </li>
	<li class="Navbar_Item_Left"> <a href={link}> <span class="Gradient"> {text}
	  </span> </a> </li>
        <li class="Navbar_Item">      Documentation          </li>
        <li class="Navbar_Item">      Help                   </li>
        <li class="Navbar_Item"> <FontAwesomeIcon icon={faCog} /> </li>
      </ul>
    </nav> 
  )
}

export default Navbar
\end{minted}

\section{Backend}

\underline{/backend/stream-token-provider/index.js}

\begin{minted}[linenos, bgcolor=lightestgray]{js}
import { StreamClient } from "@stream-io/node-sdk";

const USER_ID_LENGTH = 32

// Code to generate unique user GetStream token
async function Generate_Token(User_Id, Api_Key, Secret)
{
  const client = new StreamClient(Api_Key, Secret);

  // Create user
  const newUser = { id: User_Id };
  await client.upsertUsers([newUser]);

  // Generate token
  const token = client.generateUserToken({ user_id: User_Id });

  return token;
}

// An HTTP endpoint that allows POST requests
// with a user id given in the request. Will return
// a unique user token that can be used to begin
// video conferencing
async function Provide_Token(request, env)
{
  const { method } = request;

  // Ensure that the request is a POST
  if (method !== 'POST')
    return new Response('Method not allowed', { status: 405 });

  try {
    const { User_Id } = await request.json();

    // Ensure that a valid user id is actually provided
    if (!User_Id || User_Id.length != USER_ID_LENGTH)
      return new Response('Bad Request: Proper userId is required', { status: 400 });

    const apiKey = env.STREAM_API_KEY;
    const secret = env.STREAM_API_SECRET;

    const token = await Generate_Token(User_Id, apiKey, secret);

    return new Response(JSON.stringify({ token }), {
      headers: { "Content-Type": "application/json" },
    });

  } catch (error) {
      return new Response(`Error: ${error.message}`, { status: 500 });
  }
}

export default
{
  async fetch(request, env)
  {
    return Provide_Token(request, env);
  }
};
\end{minted}
