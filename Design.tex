\pagestyle{fancy}
\lhead{\sffamily OCR A-Level Computer Science}
\chead{\sffamily \thepage}
\rhead{\sffamily Jonathan Kasongo}
\lfoot{\sffamily Candidate number: N/A}
\rfoot{\sffamily Centre number: N/A}

\usetikzlibrary{positioning}

\chapter{Design}

\section{Breaking the problem down}
\label{sec:breakdown}

\textit
{
We now provide a simple visual decomposition of the problem at 
hand. We let DB denote our database for size reasons.
}

\begin{figure}[h]
\label{fig:decomp}
\centering

\scalebox{1.0}
{
\begin{tikzpicture}[every node/.style={scale=0.54}]

  \node[draw, rectangle, minimum width=4cm, minimum height=1cm, fill=lightestgray]
    (home) {\large Home page};

  \node[draw, rectangle, minimum width=4cm, minimum height=1cm, below=of home ]
    (login) {\large Login page};

  \node[draw, rectangle, minimum width=4cm, minimum height=1cm, below right=of home]
    (docs) {\large Documentation};

  \node[draw, rectangle, minimum width=4cm, minimum height=1cm, below left=of home]
    (help) {\large Help page};

  \node[draw, diamond, aspect=2, below=of login ]
    (new) {\large First time user?};

  \node[draw, rectangle, minimum width=4cm, minimum height=1cm, below left=of new ]
    (acc) {\large Create account};

  \node[draw, rectangle, minimum width=4cm, minimum height=1cm, below=of acc ]
    (add) {\large Add account to \textit{DB}};

  \node[draw, rectangle, minimum width=4cm, minimum height=1cm, below right=of new ]
    (back) {\large Log back in};

  \node[draw, rectangle, minimum width=4cm, minimum height=1cm, below=of back ]
    (main) {\large Main page};

  \node[draw, rectangle, minimum width=4cm, minimum height=1cm, below right=of main ]
    (join) {\large Join call page};

  \node[draw, rectangle, minimum width=4cm, minimum height=1cm, below=of join ]
    (code) {\large Enter passcode};

  \node[draw, diamond, aspect=2, minimum width=4cm, minimum height=1cm, below=of code ]
    (valid) {\large Is code valid?};

  \node[draw, rectangle, minimum width=4cm, minimum height=1cm, below left=of valid ]
    (codegood) {\large Connect to call};

  \node[draw, rectangle, minimum width=4cm, minimum height=1cm, below right=of valid ]
    (codebad) {\large Raise error};

  \node[draw, rectangle, minimum width=4cm, minimum height=1cm, below=of main ]
    (create) {\large Create call page};

  \node[draw, rectangle, minimum width=4cm, minimum height=1cm, below=of create ]
    (invite) {\large Invite others};

  \node[draw, rectangle, minimum width=4cm, minimum height=1cm, below left=of main ]
    (settings) {\large Settings page};

  \node[draw, rectangle, minimum width=4cm, minimum height=1cm, below left=of settings ]
    (video) {\large Video settings};

  \node[draw, rectangle, minimum width=4cm, minimum height=1cm, below=of settings ]
    (access) {\large Accessibility settings};

  \node[draw, rectangle, minimum width=4cm, minimum height=1cm, left=of video ]
    (audio) {\large Audio settings};

  \node[draw, rectangle, minimum width=4cm, minimum height=1cm, below=of video ]
   (save) {\large Save settings to \textit{DB}};


  \coordinate[right=2cm of code.east] (here);

  \coordinate[below=1.26cm of access.south] (a);

  \coordinate[below=1.26cm of audio.south] (b);

  \draw[black, -{Latex[length=2.5mm]}] 
    (home) -- (login);

  \draw[black, -{Latex[length=2.5mm]}] 
    (home) -| (docs);

  \draw[black, -{Latex[length=2.5mm]}] 
    (home) -| (help);

  \draw[black, -{Latex[length=2.5mm]}] 
    (login) -- (new);

  \draw[black, -{Latex[length=2.5mm]}] 
    (new) -| node[above] {Yes} (acc);

  \draw[black, -{Latex[length=2.5mm]}] 
    (new) -| node[above] {No} (back);

  \draw[black, -{Latex[length=2.5mm]}] 
    (acc) -- (add);

  \draw[black, -{Latex[length=2.5mm]}] 
    ([yshift=0.25cm]add) -| (new);
  
  \draw[black, -{Latex[length=2.5mm]}] 
    (back) -- (main);

  \draw[black, -{Latex[length=2.5mm]}] 
    (back) -- (main);

  \draw[black, -{Latex[length=2.5mm]}] 
    (main) -| (join);

  \draw[black, -{Latex[length=2.5mm]}] 
    (join) -- (code) ;

  \draw[black, -{Latex[length=2.5mm]}] 
    (code) -- (valid) ;
 
  \draw[black, -{Latex[length=2.5mm]}] 
    (valid) -| node[above] {Yes} (codegood) ;

  \draw[black, -{Latex[length=2.5mm]}] 
    (valid) -| node[above] {No} (codebad); 

  \draw[black, -{Latex[length=2.5mm]}] 
    ([xshift=0.4cm]codebad.north) -- (here) -- (code.east); 
  
  \draw[black, -{Latex[length=2.5mm]}] 
    (main) -- (create) ;

  \draw[black, -{Latex[length=2.5mm]}] 
    (create) -- (invite) ;

  \draw[black, -{Latex[length=2.5mm]}] 
    ([yshift=-0.25cm]main) -| (settings);

  \draw[black, -{Latex[length=2.5mm]}] 
    (settings) -- (access);

  \draw[black, -{Latex[length=2.5mm]}] 
    (settings) -| (video);

  \draw[black, -{Latex[length=2.5mm]}] 
    (settings) -| (audio);

  \draw[black, -{Latex[length=2.5mm]}] 
    (video) -- (save);

  \draw[black, -{Latex[length=2.5mm]}] 
    (audio.south) -- (b) -- (save.west);

  \draw[black, -{Latex[length=2.5mm]}] 
    (access.south) -- (a) -- (save.east);

\end{tikzpicture}
}

\caption{Decomposition of the problem.}
\end{figure}

\subsection*{Explaining the breakdown}

As discussed in \ref{sec:computational} decomposition can 
reduce the complexity of a system by providing clear sub-tasks
that need to be achieved in order to solve a larger more 
complicated task. Moreover this method of organising tasks 
motivates a more modular approach to the implementation of our
system; each one of the main sub-tasks is neatly and clearly 
visualised and the overall presentation shows how each
sub-task relates to the others. \\ \vspace{0.2cm}

Starting from the top of the diagram I chose to display a
home page once the user initially accesses our website. The 
home page will be primarly used to greet the user, show them 
what the web-app can do and get them to login. However from
the home page users will also be able to access the system 
documentation as well as a help page if users are having issues
with using the system. We justify the need for a homepage by 
highlighting the importance of a first impression. A 
well-designed homepage can capture the attention of the user
and encourage them to explore our web-app. If the homepage is
able to provide a good first impression we will be able to 
garner a larger userbase, and simultaneously ensure an 
excellent user experience as they move around the UI. Moreover
if our users are satisfied then our client will be too.
