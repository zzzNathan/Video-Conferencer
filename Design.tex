\pagestyle{fancy}
\rhead{\bfseries OCR A-Level Computer Science}
\chead{\bfseries \thepage}
\lhead{\bfseries Jonathan Kasongo}
\lfoot{\sffamily Candidate number: N/A}
\rfoot{\sffamily Centre number: N/A}

\usetikzlibrary{positioning}

\chapter{Design}

\section{Breaking the problem down}
\label{sec:breakdown}

\textit
{
We now provide a simple visual decomposition of the problem at 
hand. We let DB denote our database for typographical reasons.
}

\begin{figure}[h]
\label{fig:decomp}
\centering

\scalebox{1.0}
{
\begin{tikzpicture}[every node/.style={scale=0.54}]

  \node[draw, rectangle, minimum width=4cm, minimum height=1cm, fill=lightestgray]
    (home) {\large Home page};

  \node[draw, rectangle, minimum width=4cm, minimum height=1cm, below=of home ]
    (login) {\large Login page};

  \node[draw, rectangle, minimum width=4cm, minimum height=1cm, below right=of home]
    (docs) {\large Documentation};

  \node[draw, rectangle, minimum width=4cm, minimum height=1cm, below left=of home]
    (help) {\large Help page};

  \node[draw, diamond, aspect=2, below=of login ]
    (new) {\large First time user?};

  \node[draw, rectangle, minimum width=4cm, minimum height=1cm, below left=of new ]
    (acc) {\large Create account};

  \node[draw, rectangle, minimum width=4cm, minimum height=1cm, below=of acc ]
    (add) {\large Add account to \textit{DB}};

  \node[draw, rectangle, minimum width=4cm, minimum height=1cm, below right=of new ]
    (back) {\large Log back in};

  \node[draw, rectangle, minimum width=4cm, minimum height=1cm, below=of back ]
    (main) {\large Main page};

  \node[draw, rectangle, minimum width=4cm, minimum height=1cm, below right=of main ]
    (join) {\large Join call page};

  \node[draw, rectangle, minimum width=4cm, minimum height=1cm, below=of join ]
    (code) {\large Enter passcode};

  \node[draw, diamond, aspect=2, minimum width=4cm, minimum height=1cm, below=of code ]
    (valid) {\large Is code valid?};

  \node[draw, rectangle, minimum width=4cm, minimum height=1cm, below left=of valid ]
    (codegood) {\large Connect to call};

  \node[draw, rectangle, minimum width=4cm, minimum height=1cm, below right=of valid ]
    (codebad) {\large Raise error};

  \node[draw, rectangle, minimum width=4cm, minimum height=1cm, below=of main ]
    (create) {\large Create call page};

  \node[draw, rectangle, minimum width=4cm, minimum height=1cm, below=of create ]
    (invite) {\large Invite others};

  \node[draw, rectangle, minimum width=4cm, minimum height=1cm, below left=of main ]
    (settings) {\large Settings page};

  \node[draw, rectangle, minimum width=4cm, minimum height=1cm, below left=of settings ]
    (video) {\large Video settings};

  \node[draw, rectangle, minimum width=4cm, minimum height=1cm, below=of settings ]
    (access) {\large Accessibility settings};

  \node[draw, rectangle, minimum width=4cm, minimum height=1cm, left=of video ]
    (audio) {\large Audio settings};

  \node[draw, rectangle, minimum width=4cm, minimum height=1cm, below=of video ]
   (save) {\large Save settings to \textit{DB}};


  \coordinate[right=2cm of code.east] (here);

  \coordinate[below=1.26cm of access.south] (a);

  \coordinate[below=1.26cm of audio.south] (b);

  \draw[black, -{Latex[length=2.5mm]}] 
    (home) -- (login);

  \draw[black, -{Latex[length=2.5mm]}] 
    (home) -| (docs);

  \draw[black, -{Latex[length=2.5mm]}] 
    (home) -| (help);

  \draw[black, -{Latex[length=2.5mm]}] 
    (login) -- (new);

  \draw[black, -{Latex[length=2.5mm]}] 
    (new) -| node[above] {Yes} (acc);

  \draw[black, -{Latex[length=2.5mm]}] 
    (new) -| node[above] {No} (back);

  \draw[black, -{Latex[length=2.5mm]}] 
    (acc) -- (add);

  \draw[black, -{Latex[length=2.5mm]}] 
    ([yshift=0.25cm]add) -| (new);
  
  \draw[black, -{Latex[length=2.5mm]}] 
    (back) -- (main);

  \draw[black, -{Latex[length=2.5mm]}] 
    (back) -- (main);

  \draw[black, -{Latex[length=2.5mm]}] 
    (main) -| (join);

  \draw[black, -{Latex[length=2.5mm]}] 
    (join) -- (code) ;

  \draw[black, -{Latex[length=2.5mm]}] 
    (code) -- (valid) ;
 
  \draw[black, -{Latex[length=2.5mm]}] 
    (valid) -| node[above] {Yes} (codegood) ;

  \draw[black, -{Latex[length=2.5mm]}] 
    (valid) -| node[above] {No} (codebad); 

  \draw[black, -{Latex[length=2.5mm]}] 
    ([xshift=0.4cm]codebad.north) -- (here) -- (code.east); 
  
  \draw[black, -{Latex[length=2.5mm]}] 
    (main) -- (create) ;

  \draw[black, -{Latex[length=2.5mm]}] 
    (create) -- (invite) ;

  \draw[black, -{Latex[length=2.5mm]}] 
    ([yshift=-0.25cm]main) -| (settings);

  \draw[black, -{Latex[length=2.5mm]}] 
    (settings) -- (access);

  \draw[black, -{Latex[length=2.5mm]}] 
    (settings) -| (video);

  \draw[black, -{Latex[length=2.5mm]}] 
    (settings) -| (audio);

  \draw[black, -{Latex[length=2.5mm]}] 
    (video) -- (save);

  \draw[black, -{Latex[length=2.5mm]}] 
    (audio.south) -- (b) -- (save.west);

  \draw[black, -{Latex[length=2.5mm]}] 
    (access.south) -- (a) -- (save.east);

\end{tikzpicture}
}

\caption{Decomposition of the problem.}
\end{figure}

\subsection*{Explaining and justifying the breakdown}

As discussed in \ref{sec:computational} decomposition can 
reduce the complexity of a system by providing clear sub-tasks
that need to be achieved in order to solve a larger more 
complicated task. Moreover this method of organising tasks 
motivates a more modular approach to the implementation of our
system; each one of the main sub-tasks is neatly and clearly 
visualised and the overall presentation shows how each
sub-task relates to the others. \\ \vspace{0.2cm}

Starting from the top of the diagram I chose to display a
home page once the user initially accesses our website. The 
home page will be primarly used to greet the user, show them 
what the web-app can do and get them to login. However from
the home page users will also be able to access the system 
documentation as well as a help page if users are having issues
with using the system. We justify the need for a homepage by 
highlighting the importance of a first impression. A 
well-designed homepage can capture the attention of the user
and encourage them to explore the rest our web-app. If the
homepage is able to provide a good first impression we will be
able to garner a larger userbase, and simultaneously ensure an 
excellent user experience as they move around the UI. Moreover
if our users are satisfied then our client will be too. 
Documentation for the system will also be freely available to 
find on the the home page to be easily accessible for
developers. Furthermore the inclusion of the documentation 
on the home page means that developers aren't forced to create
an account just to read the documentation, saving much time for
these users. In this manner the system becomes entirely 
\textit{self-contained}, that is no other external resources 
would be necessary in order to use, maintain or update the
system. Finally in the case that users are experiencing issues
with the software a help page will also be clearly available on
the home page in order to answer FAQs as well as guide the user
through any troubleshooting. \\ \vspace{0.2cm}

The next pages require the user to first login. Upon entering 
the login page the user will be asked whether this is their 
first time using our system and if so they will be prompted to 
create an account. If it isn't the user's first time on our 
application then they will enter their username and password
and login to their account. The reason that we ask the user to 
login is because we would like each user to video conference 
with the settings that are most comfortable for them, once the
user logs in we can apply their specific accessibility 
settings that are tied to their account. Therefore in asking
the user to login before they begin video conferencing we are 
encouraging the user to take full advantage of our software by 
allowing them to first, adjust the settings to match their 
needs and requirements. \\ \vspace{0.2cm}

Once the user has logged in they will have complete access to 
the entire functionality of our web-app. I decided to seperate
the main content of the system into 3 pages, 1) Settings page,
2) Create call page, 3) Join call page. I chose to split the 
application into 3 main pages so that users won't have to 
search the app through one long overcrowded page in order to 
find what they are looking for. With this system users will be
able to quickly navigate to the page that they need and find 
what they are looking for easily on that page. Additionally, 
the concept of splitting our content onto multiple pages is 
easily scalable, if the site is updated and more content is 
added the developers can simply add a new page onto the site.
Consequently the system can rapidly expand in order to 
accommodate the growing number of user demands without
requiring a full website re-design each time an update is 
made. \\ \vspace{0.2cm}

The settings page will be where all the configurations and 
options for our system can be set/changed. It will be split
into 3 main tabs; the audio tab, the video tab and the 
accessibility tab. Each tab will hold settings related to it's
name that is, the audio tab will hold settings related volume
and sound and etc. Once a user makes some changes to any of the
settings their changes will be saved to their account and this 
data will remain on the database. The descision to split 
settings into 3 main tabs not only improves the user experience
by allowing users to find the settings they are looking for
easily but it's also in harmony with one of my client's main 
requests for this project; to have a \textit{"focus on 
simplicity"}. The descision to have settings for our system
will help each user to tailor their video conferencing
experience to fit to their personal needs. This allows us to 
create 1 system that is able to accommodate for a large 
number of people enhancing the accessibility of our platform, 
a request outlined by my client in \ref{sec:interview}. 

\newpage
\subsection{Defining the structure of the solution}

\subsubsection{Proposed system level 0 DFD}

\begin{figure}[h]
\label{fig:dfd0}
\centering
\begin{tikzpicture}

  \node[draw, rectangle, minimum width=4cm, minimum height=1cm,
	fill=lightestgray]
    (sys) {\large Proposed system};

  \node[draw, ellipse, minimum width=3cm, minimum height=1.5cm,
	below right=of sys]
    (user) {\large Users};

  \node[draw, ellipse, minimum width=3cm, minimum height=1.5cm,
	below left=of sys]
    (host) {\large Host};

  \node[draw, ellipse, minimum width=3cm, minimum height=1.5cm,
	above=of sys]
    (main) {\large Maintainers};

  \draw[black,  -{Latex[length=2.5mm]}]
	  (sys) edge["1"] (main);
  \draw[black, bend left=0.5cm, -{Latex[length=2.5mm]}]
	  (host) edge["2"] (sys);
  \draw[black, bend left=0.5cm, -{Latex[length=2.5mm]}]
	  (sys) edge["3"] (host);
  \draw[black, bend left=0.5cm, -{Latex[length=2.5mm]}]
	  (sys) edge["3"] (user);
  \draw[black, bend left=0.5cm, -{Latex[length=2.5mm]}]
	  (user) edge["4"] (sys);
  \draw[black, bend right=0.5cm, -{Latex[length=2.5mm]}]
	  (host) edge["5"] (user);
  \draw[black, bend left=0.5cm, -{Latex[length=2.5mm]}]
          (user) edge (host);

\end{tikzpicture}

\caption{Proposed system level 0 DFD diagram.}
\end{figure}

\textit{An explanation of the DFD is found below. The number 
closest to each edge refers to the number in the left column
of the explanation table below.} \vspace{-0.2cm}

%recieve feedback from users
\begin{longtblr}[
  caption={Explanation of proposed system DFD.}
]{
  colspec={lX}, hlines, row{1}={lightestgray}
}

Edge & Commentary \\

1 & {Maintainers recieve and analyse information relating to 
     the performance of the proposed system, and use the
     documentation provided with the system to suggest and 
     make any necessary changes.} \\

2 & {The host uses the proposed system in order to commence
     video conferences and to invite all those whom they would
     like to. Moreover the host will use the system to admit 
     or remove people from the video conference, as they 
     wish.} \\

3 & {The system provides the host with live audio and video 
     from all the other participants in the video conference,
     provided their microphone or camera are connected and that
     the user has activated their microphone or camera from
     their end.}\\

4 & {The user connects to video conferences via a code that 
     was given to them by the host of the meeting. They can 
     also raise their hand and tailor their experience via
     the settings page.}\\

5 & {Hosts and users are both able to communicate either via 
     their microphones or via the built-in chat box on every 
     video conference.}\\
  
\end{longtblr}

\textit{Justifying the DFD.} The DFD above allows me and my 
client to be able to clearly understand the plans for the 
proposed system, in terms of some of it's main features and 
functionalities. Furthermore the DFD diagram can be compared to
the previous system's DFD diagram to ensure that the new 
system is an adequate replacement, as discussed in
\ref{sec:currdfd}. The usage of primitive shapes and clear 
labelling ensures that the diagram is easily digestible, such 
that even those who don't understand anything about software
architecture will be able to grasp the structure of the 
proposed solution. Consequently these diagrams also work to 
improve the maintainability of the system, even new developers
with minimal programming experience will have an intuitive 
understanding of how the proposed system works and it's
overarching structure. Hence thanks to the creation of these 
diagrams we do not only equip the future maintainers of this 
system with all the necessary understanding that they require
to make changes, but we also promote a design culture of clear
and maintable software. Then if the future maintainers of this
software feel motivated to do the same we can ensure that our 
software will maximise it's longevity, through the constant 
innovation and improvement of our system namely via it's 
future maintainers.

\subsubsection{Proposed system level 1 DFD}

\subsection{\textsf Proposed algorithms}
