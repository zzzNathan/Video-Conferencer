\pagestyle{fancy}
\lhead{OCR A-Level Computer Science}
\chead{\thepage}
\rhead{Jonathan Kasongo}
\lfoot{Qualification code: H446}

\chapter{Analysis}

\section{Problem identification}

\subsection{Context}
\label{sec:1.1.1}

My client Axel Alabi has asked me to create an interactive
video conferencing application to allow others to view talks 
in realtime. The current solution is to use the \ita{Zoom} 
video conferencing application. While it is true that the 
application is technically sound and can work fine, there is a
large number of elderly users that also try to connect to the 
talks. These users often don't fully understand how to 
correctly use the application and then end up accidentally 
disturbing the conference/talk, by leaving their microphone's
on, accidentally raising their hands and so on. This makes my
client's job difficult since he is in charge of managing the 
\ita{Zoom} call. To combat this situation he would like a 
simple and user friendly video conferencing application that
provides the features needed for people to view and interact
with the conferences/talks in real time. This includes 
features like (but not limited to) audience participation,
the ability to speak to others via one's microphone and the
ability to vote on polls. The application should be
created specifically to help elderly people have a better 
experience whilst watching any conferences/talks, so may also
include extra accessibility features to ensure comfortable
viewing for all, irrespective of one's age and/or disabilities.

\subsection{Features the problem solvable through
computational methods}

Before examining the features that make this problem solvable
through computational methods, it is necessary to first 
establish a guideline for some of the basic requirements
for features that should have to be included within the 
application, based on our contextual knowledge of the problem
discussed in 
\ref{sec:1.1.1}. \vspace{0.2cm}

\begin{tblr}{
  colspec={XX},
  row{1}={lightestgray}
}
  \hline 

  \bld{Requirement} & \bld{Justification}\\

  \hline
  Real-time video footage from users webcam's. & {This is
  necessary for a \ita{video} conferencing application by 
  definition.}\\

  Real-time audio from users microphone's & {This is also 
  necessary by definition.}\\

  End to end encryption. & {Users should be able to interact
  freely without having to worry about anyone intercepting 
  their data.}\\

  A simplistic/intuitive design. & {We should ensure that the
  client's main problem is solved.}\\

  Reliability and robustness & {The application should be 
  reliable enough so that it is performant (minimal lag) and
  also has measures in place to handle any errors e.g. network
  failure.}\\

  Some accessibility features & {The application should have
  some features that enable elderly people, or disabled
  people to comfortably view the talks/conferences like 
  everyone else. }\\

  \hline
\end{tblr}

Now that we have defined some base requirements we can now
describe the computational techniques that can be applyed to 
solve the problem, and the features of the problem that make 
the technique suitable in the context of our problem.\\

\vspace{0.2cm}

\begin{tblr}{
  colspec={|XX|},
  row{1}={lightestgray}
}
  \hline

  \bld{Feature} & \bld{Justification}\\
  
  \hline

  Real-time audio/video feeds. & {To complete this part of the
  application we could apply decomposition. This problem can 
  be decomposed into multiple sub problems, for example:\\

  \vspace{0.2cm}

  1) Establish a connection to server.\\
  2) Ensure user has connected a webcam.\\
  3) Access the webcam via the relevant API.\\
  4) Send the video feed data to the server so that everyone 
  in the call may view the footage.\\

  \vspace{0.2cm}

  This idea of breaking the problem down into smaller steps 
  allows for a clear and logical approach to implementation. 
  We can also apply a generalisation to the design of the 
  application systems by making use of \ita{structured
  analysis} where we break down an entire system into smaller
  components with flow diagrams. \cite{struct}

  }\\

  \hline

  The application should be simple and user friendly. & {We 
  may apply the technique of abstraction in implementing this
  feature. By removing irrelevant information from the user
  interface we can ensure that the user only sees information
  that is relevant to them in a simple and clear manner, 
  directly achieving one of our client's requests.}\\

  \hline

  Reliability and robustness & {}\\

  \hline
\end{tblr}
