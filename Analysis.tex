\pagestyle{fancy}
\lhead{OCR A-Level Computer Science}
\chead{\thepage}
\rhead{Jonathan Kasongo}
\lfoot{Qualification code: H446}

\chapter{Analysis}

\section{Problem identification}

\subsection{Context}

My client Axel Alabi has asked me to create an interactive
video conferencing application to allow others to view talks 
in realtime. The current solution is to use the \ita{Zoom} 
video conferencing application. While it is true that the 
application is technically sound and can work fine, there is a
large number of elderly users that also try to connect to the 
conferences. These users often don't fully understand how to 
correctly use the application and then end up accidentally 
disturbing the conference/talk \footnote{From this point 
forward we will avoid using "conference/talk", and simply
replace it with "conference".}, by leaving their microphone's
on, accidentally raising their hands and so on. This makes my
client's job difficult since he is in charge of managing the 
\ita{Zoom} call. To combat this situation he would like a 
simple and user friendly video conferencing application that
provides the features needed for people to view and interact
with the conferences in real time. This includes 
features like (but not limited to) audience participation,
the ability to speak to others via one's microphone and the
ability to vote on polls. The application should be
created specifically to help elderly people have a better 
experience whilst watching any conferences, so may also
include extra accessibility features to ensure comfortable
viewing for all, irrespective of one's age and/or disabilities.

\subsection{Stakeholders}

\begin{tblr}{ll}
  \textbf{Stakeholder: } & Axel Alabi\\
  \textbf{Category: } & Client\\
\end{tblr}
\vspace{0.2cm}

\textbf{Description:} \\ 

Axel Alabi is a $22$ year old male, and is currently in charge 
of managing the video broadcasts for conferences.
He also works as a data analyst for a company specialising in
analysing geographical data. Unfortunately, managing the
broadcasts has become quite challenging because there is
often a number of elderly people who join the broadcast and
find difficulty in interacting with the broadcast. Axel would 
use the proposed the solution to not only allow everyone to be
able to access and interact with the broadcast no matter how 
much experience they have with technology. He would also use 
the solution to make his life easier and prevent people from 
disturbing the conference, allowing him to never have to worry
about manually muting individuals during the broadcast. The 
proposed solution would be appropiate to his needs because it 
simplifies his life significantly giving him less things to 
worry about and allowing him to focus solely on managing the 
broadcast. \vspace{0.2cm}

\noindent
\begin{tblr}{ll}
  \textbf{Stakeholder: } & {People aged $\sim$ 
  \hspace{-0.2cm} $50$ and over, with limited experience
  working with technology}\\
  \textbf{Category: } & Target users/audience\\
\end{tblr}
\vspace{0.2cm}

\textbf{Description: } \\

This group of users typically have limited experience working 
with technology. I aim to develop the system to be especially
suitable towards this category of people. These users will use
the software to be able to interact and access their video 
conferences in a simple and intuitive manner, without having 
to worry about the complexity and difficulty in trying to get
modern software to work correctly. The product should also 
enable any disabled ones to have a pleasant experience viewing
and interacting in video conferences. The final solution will
be appropiate to their needs as it will allow the user to be a 
part of video conferences no matter what their level of
comfortability with technology is. \vspace{0.2cm}

\noindent
\begin{tblr}{ll}
  \textbf{Stakeholder: } & IT Staff\\
  \textbf{Category: } & Support/Maintainers\\
\end{tblr}
\vspace{0.2cm}

\textbf{Description: } \\

The IT Staff would be experienced in working with technology
because of their qualifications in this field. This group of 
users should be expected to be able to update and maintain the 
system as required. To allow the staff to be able to properly 
maintain the system independently it is important to ensure
that the code is readable and clear, such that anyone reading
it can have an idea on what is going on. This will then allow
the relevant staff to make needed changes to the code without
having to try and understand what each portion of the code is
doing. This solution will be appropiate to their needs as the 
staff will now be able to tweak and change the application to 
better suite them and their situation. Furthermore the clear 
and readable code enables them to perform any necessary changes
with ease, something they could not have done previously with 
the closed-source off the shelf software they had before.
\vspace{0.2cm}

I now provide a transcript of an interview that took place 
with my client.

\begin{tcolorbox}[
  boxrule=0pt, frame empty, colback=lightgray, arc=0pt
]
  \begin{tblr}{ll}
    \textbf{Interview with Axel Alabi} & {}\\
    \textbf{Date: } \texttt{29/06/24} &
    {\hspace{-1.5cm} \textbf{Time: } \texttt{3.50pm}}
  \end{tblr}

  \vspace{0.2cm}

  \textbf{Q:} What are some essential features that should be
  required in the final application? \vspace{0.05cm}

  \textbf{A:} Well to start the app should allow users to see 
  and hear one another in real-time, there should be a focus on
  simplicity and users should be able to raise their virtual 
  "hand" to interact with the talk. \vspace{0.25cm}

  \textbf{Q:} What are some non-essential features that would
  be desirable in the final application? \vspace{0.05cm}

  \textbf{A:} The app could perhaps provide a suite of 
  accessibility features to allow disabled ones to have a 
  comfortable viewing experience. This may include closed
  captioning, volume control and a screen reader.
  \vspace{0.25cm}

  \textbf{Q:} What operating system should the application be 
  designed for? \vspace{0.05cm}

  \textbf{A:} There is no preference for operating systems.
  \vspace{0.25cm}

  \textbf{Q:} What are the software requirements? 
  \vspace{0.05cm}

  \textbf{A:} It should be a web-based application. Any 
  suitable mainstream programming language is fine as long as 
  the code is clear enough for me and the other IT staff to 
  understand.
  \vspace{0.25cm}

  \textbf{Q:} What are the security requirements?
  \vspace{0.05cm}

  \textbf{A:} There should be some form of end to end 
  encryption to ensure that hackers or others cannot access the  video feeds. There should also be some kind of username and 
  password system in order to enter a call. Passwords should 
  also be of a good strength e.g. at least 1 symbol, capital
  and lowercase letters
  \vspace{0.25cm}

  \textbf{Q:} How will the new system benefit you? 
  \vspace{0.05cm}

  \textbf{A:} This new system will ensure that all video 
  conferences I am in charge of managing will run much 
  smoother, not only giving me more time to work on other 
  essential tasks but also providing a better viewing
  experience for all.

\end{tcolorbox}

\subsection{Features that make the problem solvable via
computational methods}

A table of features, their justifications and how they are 
amenable to be solved via computational methods is now
provided. 

\begin{longtblr}[
  caption={Features and their justifications},
  label={tblr:features}
]{
  colspec={|XX|},
  row{1}={lightestgray},
  hlines,
}

  \bld{Feature} & \bld{Justification}\\
  
  Real-time audio/video feeds. & {To justify this feature, note
  that \textit{video} and \textit{audio} feads should be 
  nessecary for a \textit{video} chat application. It is also 
  explicitly requested for by my client. To complete this part
  of the application we could apply decomposition. This
  problem can be decomposed into multiple sub problems, 
  for example:\\

  \vspace{0.2cm}

  1) Establish a connection to server.\\
  2) Ensure user has connected a webcam.\\
  3) Access the webcam via the relevant API.\\
  4) Send the video feed data to the server/host so that
  everyone 
  in the call may view the footage.\\

  \vspace{0.2cm}

  This idea of breaking the problem down into smaller steps 
  allows for a clear and logical approach to implementation. 
  }\\

  The application should be simple and user friendly. & {This
  is again one of the explicit requests made by my client. We 
  may apply the technique of abstraction in implementing this
  feature. By removing irrelevant information from the user
  interface we can ensure that the user only sees information
  that is relevant to them in a simple and clear manner, 
  directly achieving one of our client's requests.}\\

  Usernames and passwords to enter a call & {Client has 
  explicitly requested this feature, as it ensures that nobody
  who wasn't invited to the conference can join it. This 
  problem is suitable to algorithmic thinking, we can utilise
  a sequence of steps in order to ensure that each call has 
  it's own distinct password, and to ensure that when the user
  types in a password they are admitted into the correct 
  call. For example:\\
  
  \vspace{0.2cm}
  1) Upon creation of the call the user is given a randomly 
  generated string to be used as a password for others to 
  join.\\

  2) This passcode is stored in a database on a server.\\

  3) When a user wants to join a call the password they enter 
  is checked against entries into that database, and the user
  is connected if a matching password is found.\\
  \vspace{0.2cm}

  Algorithmic thinking allows us to solve this problem using 
  set procedures and rules exactly how a computer would have to  solve this problem. Thinking in this way therefore allows us
  to easily transition from ideas to code.}\\

  Screen reader & {During our interview client suggested this
  to be one of the accessibility features that could be
  implemented in the final application. It could help those 
  who have poor eyesight to be able to navigate the application
  independently without the need for assistance from others. 
  This problem could be amenable to both pattern recognition 
  and/or abstraction. In the event that I implement my own 
  screen reader I could notice that some words are very likely
  to occur in sequence. For example the word "how" may be 
  followed by "are you?" frequently enough for us to prepare 
  audio for the phrase "are you?" once we recognise that we 
  have just read the word "how". This improves the performance
  of the system overall as less resources have to be expended
  on screen reading. In the event where I use an
  existing API for this task I could apply abstraction to 
  remove any unnecessary options or configurations to the 
  screen reader leaving only essential options like reading 
  pace and accent, available to the user. This improves user
  experience as they are no longer overwhelmed by a multitude
  of choices, leading to less stress and a more comfortable 
  experience. \cite{overchoice}}

\end{longtblr}

\subsection{Research}
