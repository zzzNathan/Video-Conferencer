\pagestyle{fancy}
\lhead{OCR A-Level Computer Science}
\chead{\thepage}
\rhead{Jonathan Kasongo}
\lfoot{Qualification code: H446}

\chapter{Analysis}

\section{Problem identification}

\subsection{Context}
\label{sec:context}

My client Axel Alabi has asked me to create an interactive
video conferencing application to allow others to view talks 
in realtime. The current solution is to use the Zoom 
video conferencing application. While it is true that the 
application is technically sound and can work fine, there is a
large number of elderly users that also try to connect to the 
conferences. These users often don't fully understand how to 
correctly use the application and then end up accidentally 
disturbing the conference/talk \footnote{From this point 
forward we will avoid using "conference/talk", and simply
replace it with "conference".}, by leaving their microphone's
on, accidentally raising their hands and so on. This makes my
client's job difficult since he is in charge of managing the 
Zoom call. To combat this situation he would like a 
simple and user friendly video conferencing application that
provides the features needed for people to view and interact
with the conferences in real time. This includes 
features like (but not limited to) audience participation,
the ability to speak to others via one's microphone and the
ability to vote on polls. The application should be
created specifically to help elderly people have a better 
experience whilst watching any conferences, so may also
include extra accessibility features to ensure comfortable
viewing for all, irrespective of one's age and/or disabilities.

\subsection{Stakeholders}

\begin{tblr}{ll}
  \textbf{Stakeholder: } & Axel Alabi\\
  \textbf{Category: } & Client\\
\end{tblr}
\vspace{0.2cm}

\textbf{Description:} \\ \vspace{0.05cm}

Axel Alabi is a $22$ year old male, and is currently in charge 
of managing the video broadcasts for conferences. He also
works as a data analyst for a company specialising in analysing
geographical data, and has experience working in computer 
science. Unfortunately, managing the broadcasts has become 
quite challenging because there is often a number of elderly
people who join the broadcast and find difficulty in
interacting with the broadcast. Axel would use the proposed 
solution as a replacement for the Zoom videoconferences he uses
currently. When asked to manage/set up a videoconference by 
his company he would go onto his web browser onto my website,
and begin a conference. He would be given a unique entrance
code and then he would give this code to the people who are to 
be invited to the videoconference. The solution would provide a
simpler, more intuitive and more accessible alternative to
Zoom. The proposed solution would be appropiate to Axel and his
needs because the new system will be easier to use for his 
audience and consequently he will have less trouble in managing
the call. Instead of him having to worry about muting the 
audience's microphone's and responding to any technical
difficulties/issues, he would be able to run the conference 
smoothly. This would improve his life greatly as he would be 
able to now focus solely on ensuring that video and audio is 
clear during the broadcast. In removing some of Axel's issues
that he faces in his job, he should become more content and 
calmer with his work life, adequately showing how this solution
will be appropiate to his needs.

\vspace{0.2cm}

\noindent
\begin{tblr}{ll}
  \textbf{Stakeholder: } & {People aged $\sim$ 
  \hspace{-0.2cm} $50$ and over, with limited experience
  working with technology}\\
  \textbf{Category: } & Target users/audience\\
\end{tblr}
\vspace{0.2cm}

\textbf{Description: } \\ \vspace{0.05cm}

This group of users typically have limited experience working 
with technology. I aim to develop the system to be especially
suitable towards this category of people. These users will use
the software to be able to interact and access their video 
conferences in a simple and intuitive manner, without having 
to worry about the complexity and difficulty in trying to get
modern software to work correctly. The product should also 
enable any disabled ones to have a pleasant experience viewing
and interacting in video conferences. The final solution will
be appropiate to their needs as it will allow the user to take
part in video conferences simply no matter what their level of
comfortability with technology is. Futhermore the solution will
also be specifically tailored towards this group of people, 
that is the final product will be designed in such a manner so
that it feels easy to grasp for those users with limited 
experience with technology, immediately rectifiying the 
shortcomings of the previous system. \vspace{0.2cm}

\noindent
\begin{tblr}{ll}
  \textbf{Stakeholder: } & IT Staff\\
  \textbf{Category: } & Support/Maintainers\\
\end{tblr}
\vspace{0.2cm}

\textbf{Description: } \\ \vspace{0.05cm}

The IT Staff would be experienced in working with technology
because of their qualifications in this field. This group of 
users would be expected to be able to update and maintain the 
system as required. To allow the staff to be able to properly 
maintain the system independently it is important to ensure
that the code is readable and clear, such that anyone reading
it can have an idea on what is going on. This will then allow
the relevant staff to make needed changes to the code without
having to try and understand what each portion of the code is
doing. This solution will be appropiate to their needs as the 
staff will now be able to tweak and change the application to 
better suite them and their situation. Furthermore the clear 
and readable code enables them to perform any necessary changes
with ease, something they could not have done previously with 
the closed-source off the shelf software they had before.

\subsection{Research}
\label{sec:research}

% Researched the problem in depth looking at existing solutions
% to similar problems, identifying and justifying suitable 
% approaches based on this research.

We begin our research by observing solutions to similar 
problems. Then justifications of suitable approaches are given
based on the existing solutions.

\subsection*{Zoom}

\subsection*{Accessibility features}

Zoom is a popular closed-source video chat application,
developed by Zoom video communications. From their website 
\cite{zoom} they implement 5 major features to ensure that 
\textit{"Zoom is for everyone"}. 

\begin{itemize}
  \item Live transcriptions
  \item Automatic closed captioning
  \item Customisation of font sizes
  \item Keyboard shortcuts
  \item Screen reader support
\end{itemize}

Some advantages of this software include the fact that there
is a good amount of accessibility features able to help a wide
variety of people. For instance for those who have limited 
mobility keyboard shortcuts can be set up to conveniently 
perform common tasks, whilst those who have trouble hearing
well can enable closed captioning during the meeting.

\vspace{0.2cm}

However there are still a number of limitations with the way 
these features are implemented. Automatically generated 
closed captioning is unfortunately only available in english,
and may have varying levels of accuracy depending on external
factors such as background noise, speaker's clarity and 
profiency in spoken english.

\vspace{0.2cm}

Zoom proposes a solution to the problem of exculsivity when in 
the context of video conferencing applications. Instead of only
designing the application to be usable for one group of people
they aim to instead tailor it to cater towards 
\textit{everyone}. I believe that the wide range of
accessibility features is a big advantage of the system, and 
this could motivate the descision to implement a similar set
of features in my application. However while a good number of 
features is appreciable, it is also necessary to ensure that 
these features are simple to find and to use for an optimal 
user experience. This is justified by the fact that my client
is specifically requesting an application for those who aren't
comfortable with modern technology. To solve this problem I 
could perhaps implement a built-in tutorial that demonstrates 
how to use the accessibility features to teach the user how 
they can use the application to it's fullest potential.

\subsection*{Skype}

\subsection*{Accessibility features}

Skype is a proprietary messaging and video chat application
developed by Skype technologies a subsidiary of Microsoft 
\cite{skype}. From Microsoft's support webpage Skype for 
Windows 8 and above has the 3 following key accessibility 
features.

\begin{itemize}
  \item Narrator screen reader
  \item High contrast colour settings
  \item Magnifier
\end{itemize}

These features have the advantage of being especially 
accessible for blind people or for those with low vision. The 
combination of high contrast colours along with a magnifier 
and/or screen reader ensures that low vision users can still 
make use of Skype, independently. Skype also doubles up to be
an instant messaging platform, permitting users to have all 
their conversations and other communications in 1 place.

\vspace{0.2cm}

Whilst the features mentioned are very beneficial for those 
with low vision, there is no true support for people who are
hard of hearing or have low mobility. This set of features is 
one-dimensional only catering to 1 group of less abled people.

\vspace{0.2cm}

Skype proposes a solution to \textit{"help people with 
disabilities navigate and control their device as well as get
better access to online content."} 
\footnote{Quote from \url{https://support.microsoft.com/en-gb/skype/what-accessibility-features-are-available-for-skype-89c34c52-f463-437a-b3be-2ea114c5de13}}
The solution from Skype allows for disabled ones especially 
those with reduced vision to benefit from Skype the most, as 
previously discussed. Whilst Skype offers a good set of
features for the disabled, the features that they mention on 
their website are all already either implemented on most 
popular operating systems, or can be installed as browser 
extensions with a few clicks. Therefore I believe that there is
insufficient justificaiton to take the time to implement any 
of the features that Skype implements, and will not be 
implementing any of those features.

\vspace{0.2cm}

To obtain a better understanding of the nature of the problem
and what features should be implemented in the final solution,
I decided to collect some qualititative data through an 
interview with my client. This data should allow me to have an
insight into what the final solution should look like based
on my client's requirements and desires respectively.

\begin{tcolorbox}[
  boxrule=0pt, frame empty, colback=lightestgray, arc=0pt,
  breakable, colframe=white
]
  \begin{tblr}{ll}
    \textbf{Interview with Axel Alabi} & {}\\
    \textbf{Date: } \texttt{29/06/24} &
    {\hspace{-1.5cm} \textbf{Time: } \texttt{3.50pm}}
  \end{tblr}

  \vspace{0.2cm}

  \textbf{Q:} What are some of the limitations of the current
  system used for video conferencing? \vspace{0.05cm}

  \textbf{A:} It tends to be difficult for people who aren't 
  experienced with technology to properly interact in the 
  conferences. Often times participants will accidently turn 
  their microphone on or are unable to turn their microphone
  on when the speaker invites them to. \vspace{0.25cm}

  \textbf{Q:} What are some essential features that should be
  required in the final application? \vspace{0.05cm}

  \textbf{A:} Well to start the app should allow users to see 
  and hear one another in real-time, there should be a focus on
  simplicity and users should be able to raise their virtual 
  "hand" to interact with the talk. \vspace{0.25cm}

  \textbf{Q:} What are some non-essential features that would
  be desirable in the final application? \vspace{0.05cm}

  \textbf{A:} The app could perhaps provide a suite of 
  accessibility features to allow disabled ones to have a 
  comfortable viewing experience. This may include closed
  captioning, volume control and a screen reader.
  \vspace{0.25cm}

  \textbf{Q:} What operating system should the application be 
  designed for? \vspace{0.05cm}

  \textbf{A:} There is no preference for operating systems.
  \vspace{0.25cm}

  \textbf{Q:} What are the software requirements? 
  \vspace{0.05cm}

  \textbf{A:} It should be a web-based application. Any 
  suitable mainstream programming language is fine as long as 
  the code is clear enough for me and the other IT staff to 
  understand.
  \vspace{0.25cm}

  \textbf{Q:} What are the security requirements?
  \vspace{0.05cm}

  \textbf{A:} There should be some form of end to end 
  encryption to ensure that hackers or others cannot access the
  video feeds. There should also be some kind of username and 
  password system in order to enter a call. Passwords should 
  also be of a good strength e.g. at least 1 symbol, capital
  and lowercase letters.
  \vspace{0.25cm}

  \textbf{Q:} How will the new system benefit you? 
  \vspace{0.05cm}

  \textbf{A:} This new system will ensure that all video 
  conferences I am in charge of managing will run much 
  smoother, not only giving me more time to work on other 
  essential tasks but also providing a better viewing
  experience for all.

\end{tcolorbox}

From this interview, I can see that my client has some features
that should definitely be included in the solution: real-time 
audio and video as well as the ability to raise and lower one's
virtual hand. Furthermore I believe that security should also 
be very important when designing the final solution. This 
should be a requirement because we don't want any of our users
to have their important data stolen during their video
conference. \vspace{0.2cm}

Bearing this in mind we should also think about suitable 
approaches to implementing a system with these features. In 
order to implement the real-time video/audio I could make use
of the WebRTC API from Google, which would enable me to 
establish secure peer to peer connections from the browser. 
Not only does this approach allow us to establish audio and 
video, but it also ensures that these connections are secure 
through the use of signalling servers. Signalling servers are
servers that manage connections between peers. To understand 
signalling servers we will go through an example. Consider 2 
users, Alice and Bob that want to have a video call. Alice 
creates an offer for Bob to connect. Consequently WebRTC 
creates a session description protocol (SDP) object. The SDP 
object holds information like media types, name of the
session and the video codec being used. This data is then 
saved to a \textit{signalling} server. Bob then reads this SDP
offer from the server and WebRTC creates a SDP answer and
writes this to the server. Alice and Bob have now established 
a peer to peer connection. In essence, signalling allows for 
users to exchange the metadata of their connection through the
WebRTC API. To justify usage of the WebRTC API, I believe it 
is first necessary to explain why I rejected the idea of
implementing an API from scratch. Unfortunately
implementations of fully functioning and secure peer to peer
API's aren't trivial at all. The time it would take to fully
understand how to implement peer to peer connections using 
VP9 packetizers \cite{vp9} and SHA-256 cryptography for data
security would simply be too time-consuming and cannot be
justified when fully functioning, tested and performant 
implementations already exist. The next 
portion of the justification covers why did I choose the
WebRTC API over other API's? WebRTC is from Google and it is 
well known that Google is a credible technology company. 
Therefore it is sensible to assume that their API is of 
highest quality publicly available right now. However, there
are alternatives to WebRTC like: VideoSDK, Twilio and 
MirrorFly. Whilst it can be acknowledged that these API's could
potentially be a better fit for my project in terms of
performance, features and simplicity, all of these alternatives
are paid for. I don't think it would be sensible to pay for
commercial API's when there are free one's like WebRTC
available that will work just fine.

\vspace{0.2cm}

When discussing security it is also important to discuss the 
username and password management system that may be implemented
since it is one of my client's requests. In order to design a 
system that has secure password management it will be 
beneficial to study how other industry applications have chosen
to solve the problem of managing user passwords. I chose to 
study Bitwarden, a freemium open source password management 
service. The following information was found on their 
architecture webpage \cite{Bitwarden}. As per their website 
they use the \textit{"Command and Query Responsibility 
Segregation (CQRS) pattern"}. The CQRS model developed by 
Microsoft separates reads and writes into different models. 
\textit{Commands} are used to write to the database, whilst 
\textit{queries} are used to read from the database. Each 
command and query has one \emph{single} responsibility and 
should be based on actions rather than operations on data. For
instance Microsoft give the example rather than use the
data-based command: \code{SET ReservationStatus to Reserved}
we should prefer to use the action-based command: \\
\code{Book\_Hotel\_Room()} instead. Some of the benefits of 
this model include: 

\begin{itemize}
  \item Security. It's easier to ensure that only people of authority are performing reads and writes to the database if 
they're separated.

  \item Separation of concerns. Splitting the read and write components makes the system more modular and more maintainable. 

  \item Simplified commands. Instead of directly manipulating data commands should be based on the task they try to 
achieve.
\end{itemize}

Bitwarden's servers use the MSSQL database and makes automated
nightly backups to ensure data is protected. Furthermore the 
company uses \textit{zero-knowledge encryption} so that the 
company cannot see it's users data. Through this kind of 
encryption the company is able to have it's users complete 
trust as they are sure that their data is safe and cannot be
read by anyone other than them. Our application could 
definitely implement the CQRS model. The zero-knowledge 
encryption could perhaps be implemented using an existing API,
this is because the algorithms needed to implement 
zero-knowledge encryption include SHA-256 as well as 
AES-CBC 256 bit, both algorithms which require large amounts of 
higher level mathematics.

\subsection{Essential features and their justifications}

\textbf{Real-time audio/video feeds} \\ \vspace{0.1cm}

This will enable multiple users to connect to each other via
their browsers and view each other's webcams, as well as hear
each other in real-time. The justification for this feature is
that it is explicitly requested for by my client in our
interview, furthermore we don't have a \textit{video}
conferencing application if we cannot actually see other 
people's video feeds on our application.

\vspace{0.2cm}

\textbf{Raising one's virtual hand} \\ \vspace{0.1cm}

This feature will allow people to be able to interact with the
person giving the talk as if they were present in real life.
The virtual hand will be made visible to all participants in 
the conference so that the speaker/host will be able to ask 
the audience members to speak when appropiate. This feature
is sufficiently justified because of my client's specific 
request to implement it as a feature in the interview. 
Furthermore the implementation of this feature will help the 
migration from Zoom to our new platform be familiar as this 
feature is also found on Zoom.

\vspace{0.2cm}

\textbf{Designation of a host} \\ \vspace{0.1cm}

This feature will allow the creator of the conference to 
assign 1 or multiple people to be the \textit{host} of the 
conference. That means that these people will be in charge of 
managing the conference and will control who to admit into the
conference call, who to unmute and who to remove from the call,
and will be able to lower other's virtual hand. The
justification of this feature is that I believe there should 
always be someone of some sort of authority to coordinate and 
manage these conferences. This is very common in real life also
because without management and coordination people would be 
clueless, and anarchy would run rampant. This phenomenon has
been seen in other video chat applications more specifically,
on Zoom \footnote{See 
\url{https://en.wikipedia.org/wiki/Zoombombing} for more 
information}, and in order to provide the best experience for
our users it is vital that events like this aren't allowed to
happen.

\vspace{0.2cm}

\textbf{Usernames and passwords} \\ \vspace{0.1cm}

Each user will be able to choose their unique username so 
that users can easily identify one another upon joining a call.
Furthermore each user can set their own password so that their
account is protected and others cannot pretend to be them. 
Assigned to each user account will be their saved settings and
options, so that if a user has a specific configuration of 
settings that are adapted for their needs they don't have to 
set these up each time they log onto a conference. Passwords 
will be made to fit some set of requirements to ensure that 
the password is of sufficient strength. The justification of 
this feature is clearly evident enough for it to be considered
essential to the final solution, and moreover the feature was
also requested during my interview with the client.

\vspace{0.2cm}

\textbf{WebRTC for peer to peer connections} \\ \vspace{0.1cm}

We will make use of the WebRTC API, in order to establish peer 
to peer connections between all major web browsers, like
Google, Firefox and Safari. This decision was fully justified 
in \ref{sec:research}. The API will enable users to transfer
their video and audio data between one another securely and
efficiently, over the internet.

\subsection{Problems, issues and limitations}

\textbf{}
