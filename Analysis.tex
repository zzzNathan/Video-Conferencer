\pagestyle{fancy}
\lhead{OCR A-Level Computer Science}
\chead{\thepage}
\rhead{Jonathan Kasongo}
\lfoot{Qualification code: H446}

\chapter{Analysis}

\section{Problem identification}

\subsection{Context}
\label{sec:1.1.1}

My client Axel Alabi has asked me to create an interactive
video conferencing application to allow others to view talks 
in realtime. The current solution is to use the \ita{Zoom} 
video conferencing application. While it is true that the 
application is technically sound and can work fine, there is a
large number of elderly users that also try to connect to the 
talks. These users often don't fully understand how to 
correctly use the application and then end up accidentally 
disturbing the conference/talk, by leaving their microphone's
on, accidentally raising their hands and so on. This makes my
client's job difficult since he is in charge of managing the 
\ita{Zoom} call. To combat this situation he would like a 
simple and user friendly video conferencing application that
provides the features needed for people to view and interact
with the conferences/talks in real time. This includes 
features like (but not limited to) audience participation,
the ability to speak to others via one's microphone and the
ability to vote on polls. The application should be
created specifically to help elderly people have a better 
experience whilst watching any conferences/talks, so may also
include extra accessibility features to ensure comfortable
viewing for all, irrespective of one's age and/or disabilities.

\subsection{Stakeholders}

\begin{tblr}{ll}
  \textbf{Stakeholder: } & Axel Alabi\\
  \textbf{Category: } & Client\\
\end{tblr}
\vspace{0.2cm}

\textbf{Description:} \\ 

Axel Alabi is a $22$ year old male, and is currently in charge 
of managing the video broadcasts for conferences and/or talks.
He also works as a data analyst for a company specialising in
analysing geographical data. Unfortunately, managing the
broadcasts has become quite challenging because there is
often a number of elderly people who join the broadcast and
find difficulty in interacting with the broadcast. Axel would 
use the proposed the solution to not only allow everyone to be
able to access and interact with the broadcast no matter how 
much experience they have with technology. He would also use 
the solution to make his life easier and prevent people from 
disturbing the conference, allowing him to never have to worry
about manually muting individuals during the broadcast. The 
proposed solution would be appropiate to his needs because it 
simplifies his life significantly giving him less things to 
worry about and allowing him to focus solely on managing the 
broadcast. \vspace{0.2cm}

\noindent
\begin{tblr}{ll}
  \textbf{Stakeholder: } & People aged $\sim 40$ and over\\
  \textbf{Category: } & Target users/audience\\
\end{tblr}
\vspace{0.2cm}

\textbf{Description: } \\

This group of users typically have limited experience working 
with technology. I aim to develop the system to be especially
suitable towards this category of people. These users will use
the software to be able to interact and access their video 
conferences in a simple and intuitive manner, without having 
to worry about the complexity and difficulty in trying to get
modern software to work correctly. The final solution will be
appropiate to their needs as it will allow the user to access
video conferences no matter what their experience level with 
technology is.\vspace{0.2cm}

\noindent
\begin{tblr}{ll}
  \textbf{Stakeholder: } & IT Staff\\
  \textbf{Category: } & Support/Maintainers\\
\end{tblr}
\vspace{0.2cm}

\textbf{Description: } \\

The IT Staff would be experienced in working with technology
because of their qualifications in this field. This group of 
users should be expected to be able to update and maintain the 
system as required. To allow the staff to be able to properly 
maintain the system independently it is important to ensure
that the code is readable and clear, such that anyone reading
it can have an idea on what is going on. This will then allow
the relevant staff to make needed changes to the code without
having to try and understand what each portion of the code is
doing.\vspace{0.2cm}

I now provide a transcript of an interview that took place 
with my client.

\begin{tcolorbox}[
  boxrule=0pt, frame empty, colback=lightgray, arc=0pt
]
  \begin{tblr}{ll}
    \textbf{Interview with Axel Alabi} & {}\\
    \textbf{Date: } \texttt{29/06/24} &
    {\hspace{-1.5cm} \textbf{Time: } \texttt{3.50pm}}
  \end{tblr}

  \vspace{0.2cm}

  \textbf{Q:} What are some essential features that should be
  required in the final application? \vspace{0.05cm}

  \textbf{A:} Well to start the app should allow users to see 
  and hear one another in real-time, there should be a focus on
  simplicity and users should be able to raise their virtual 
  "hand" to interact with the talk. \vspace{0.25cm}

  \textbf{Q:} What are some non-essential features that would
  be desirable in the final application? \vspace{0.05cm}

  \textbf{A:} The app could perhaps provide a suite of 
  accessibility features to allow disabled ones to have a 
  comfortable viewing experience. This may include closed
  captioning, volume control and a screen reader.
  \vspace{0.25cm}

  \textbf{Q:} What operating system should the application be 
  for? \vspace{0.05cm}

  \textbf{A:} There is no preference for operating systems.
  \vspace{0.25cm}

  \textbf{Q:} What are the software requirements? 
  \vspace{0.05cm}

  \textbf{A:} It should be a web-based application. Any 
  suitable mainstream programming language is fine.
  \vspace{0.25cm}

  \textbf{Q:} What are the security requirements?
  \vspace{0.05cm}

  \textbf{A:} There should be some form of end to end 
  encryption to ensure that hackers or others cannot access the  video feeds. There should also be some kind of username and 
  password system in order to enter a call. Passwords should 
  also be of a good strength e.g. at least 1 symbol, capital
  and lowercase letters
  \vspace{0.25cm}

\end{tcolorbox}

\subsection{Features that make the problem solvable via
computational methods}

\begin{tblr}{
  colspec={|XX|},
  row{1}={lightestgray}
}
  \hline

  \bld{Feature} & \bld{Justification}\\
  
  \hline

  Real-time audio/video feeds. & {To justify this feature, note
  that \textit{video} and \textit{audio} feads should be 
  nessecary for a \textit{video} chat application. It is also 
  explicitly requested for by my client. To complete this part
  of the application we could apply decomposition. This
  problem can be decomposed into multiple sub problems, 
  for example:\\

  \vspace{0.2cm}

  1) Establish a connection to server.\\
  2) Ensure user has connected a webcam.\\
  3) Access the webcam via the relevant API.\\
  4) Send the video feed data to the server/host so that
  everyone 
  in the call may view the footage.\\

  \vspace{0.2cm}

  This idea of breaking the problem down into smaller steps 
  allows for a clear and logical approach to implementation. 
  }\\

  \hline

  The application should be simple and user friendly. & {We 
  may apply the technique of abstraction in implementing this
  feature. By removing irrelevant information from the user
  interface we can ensure that the user only sees information
  that is relevant to them in a simple and clear manner, 
  directly achieving one of our client's requests.}\\

  \hline

  Reliability and robustness & {}\\

  \hline
\end{tblr}
